\section{叶韵致}
背景:
\begin{itemize}
\item GPA: 86/100
\item TOEFL: 103 (R29+L28+W27+S19)
\item 无研究、无牛推
\item 实习:外企实习六个月(eBay水实习)
\end{itemize}\par
申请结果:
\begin{itemize}
\item AD: Columbia, Rice, UCI, Stony Brook, TAMU, UC-Boulder, NEU
\item Rej: Yale, Upenn, UCLA, UCSD, UMASS, UFL, Gatech, Dartmouth, Brown
\item Pending: USC
\end{itemize}
最后去了Columbia。\par
一句话概况起来,这次申请不算成功,也不算失败吧。最终去的学校也算是比较满意了。
下面详细介绍一下我的申请总结和教训吧。
\subsection{关于GRE/TOEFL}
就我的申请结果而言,我觉得GRE/TOEFL过线就行了。尤其是GRE,感觉大部分学校都不怎么重视GRE的成绩,一般320是一条线,过了320的话GRE就不会拖你后腿了,不过超过330的GRE是一个亮点,尤其是对个别比较看重GRE成绩单学校(比如yale,录取的人都有超高的GRE成绩)。TOEFL过100,应该算是硬性要求吧,虽然很多学校的招生网站上说,他们要求的托福底线是85,87什么的,其实他们实际录取的情况都比这个要高很多,一般观点是100分是一条线。其实过100并不难,屌丝如我的英语四级没上550,六级没上500的学渣,努努力还是能上100的。\par
至于如何准备GRE和TOEFL考试,我就不多说了,一者,我两项成绩平平,并没有什么比较牛逼的应试经验,二者,网上很多牛人分享的经验,一搜一大把,而且我觉得他们都说得挺好的,我就不重复了。不过GRE和TOEFL的确是需要花时间和经历的,当你过了这两道坎,申请也成功了一半了。
\subsection{关于GPA}
GPA绝对是申请里面的重头戏。除开各项软实力,单看这几项硬实力,GPA远大于TOEFL、GRE。GPA的重要性自是不容忽视的,平均分85是一条线,越往上走优势越大,到了90的平均分是具有很强的竞争力的。所有课,不管是选修课还是必修课,都要一个劲往高了考,因为申请的时候,对方学校不会管你是不是选修课,统统加在一起算你的GPA,所以不要因为是选修课就无所谓。当然水课和好课的权衡要靠你自己了,水课容易得高分,但是学不到什么东西;好课可能意味着较低的成绩(牛人请无视),但是能够学到实实在在的东西。 所有课尽量保持在80分以上,没上80的话就重修吧(如果是毛概,马哲什么的,还是算了,随他吧)。不过重修还是要量力而行,自己把握课程压力,不然得不偿失。
\subsection{文书准备}
软实力其实是我很想在这个总结里面讲的内容,因为我的申请就差在了这个方面上。软实力包括很多,科研,交流经历,实习经历,我觉得推荐信也算是软实力。\par

科研,直接衡量标准是论文。论文虽然对于PHD申请者来说更为重要,但是2013FALL的经验告诉我,有了论文,Master申请也会平步青云。因为在申请的时候你会发现,你的竞争者们很多手握一两篇论文,和你一起抢master的坑,所以可以搞搞研究还是尽量参与到研究中去,一方面可以深度了解某一个具体的领域,一方面有机会发论文,再一方面还有可能套到牛推。\par

交换经历也很重要的!有交换经历的同学,一方面可以拿到外国教授的推荐信,另一方面可以获得国外学校的成绩单。对于你要申请的学校而言,他们并不十分了解中国的学校的情况,你的成绩好坏对于他们来说并没有直观的认识,如果你有国外大学的成绩单,这样他们对你的学习能力就有数了。而且国外本科拿个很高的GPA对于你们来说应该不太难。\par

抛开其他的方面不说,功力地分析实习经历,如果你是去MRSA或者中科院什么的去实习的话应该会加分不少,其他地方的实习经历对于出国申请帮助不大。所以,如果功利一点,能去那两个地方就去,去不了可以留下来跟着学校老师做研究(当然,去清华北大做研究也可以,还有海外研究也是一个非常不错的选择)。不功利的讲,去创业公司实习收获最大,当然也最辛苦。\par

下面是我想介绍的整个申请过程中了解到的一些学校的信息,希望这些信息对以后的学弟学妹有帮助吧。

\subsection{学校信息}
\textbf{UCLA}\par
综合排名:24 专业排名:14 申请难度:***** \par

UCLA是曾经的dream school啊。但是看了他给的AD的申请者的背景,顿时就萎了。各种论文哥,各种高GPA,申请难度和Stanford有的一拼。当然UCLA如此傲娇也是有他傲娇的资本的。地处加州洛杉矶,地理位置的优势为他加分不少。加州对于CS专业来说毋庸置疑是块风水宝地。各个IT巨擘还有数不胜数的创业小公司,为加州的CS行业推波助澜,形成了一片祥和的景致。因此这也为加州系列的各大学提供了非常强劲的竞争资本。\par

当然,如果只是地理位置好,UCLA还不能如此傲娇,最主要的还是要依仗他的计算机实力,UCLA的计算机科学实力不容小觑,从排名就可见一斑,尤其是他的AI和ML方向,非常之强(AI和ML也是现在很火的方向之二)。\par

所以,要拿到UCLA的AD,并不容这么容易,发几篇论文,搞高GPA是很有必要的。\par

\textbf{UMASS}\par
综合排名:97 专业排名:20 申请难度:**** \par

UMASS 招master很少,基本只招博士。Master招过去基本也是搞研究,而且很有可能有奖。不过招人很少很少,它的AI特别强,貌似排名前十。UMASS也是很看重申请者的研究背景的,能有papper,就会离UMASS的master AD更近一步。\par

UMASS的优点也特别明显,专业排名不错,20名,和UCLA一样,AI特别强。另外地处波士顿附近,就业机会也比较多,当然波士顿这边的竞争也很激烈,哈佛,MIT都在这边扎堆,还有东北这种针对就业的职业培训大学。不过UMASS的计算机实力在那儿,自然是不怕竞争的。除此之外,UMASS还有一个优点,那就是学费特别便宜,具体是多少不太记得了,印象中特别便宜,同时还有很大的几率获得奖学金,那自然又在便宜的基础上省了不少。\par

揣测UMASS的招生。UMASS今年发AD的速度很慢,都是隔几天发一个AD出来,然后又隔几天再发一个,而且AD和REJ一起发。我比几个同学申请得早,但是最后出结果出的比他们晚,很有可能是我在ps里面强调自己非常喜欢AI,然后他们看到比较匹配他们的优势方向,就对我判了死缓(不过最后还是被一枪毙了),所以我猜想他们一定是很细致很细致地研究申请者的申请材料。因此在准备他家的申请材料的时候一定要细致,最好做到没有纰漏。\par

\textbf{Brown}\par
综合排名:15 专业排名:20 申请难度:**** \par

著名藤校之一。招人也不太多,难度大,不过学校不错。Brown出了名的对学生好,据说研究生每人分配了办公室,各种资源非常丰富。另外学校的教学质量也特别好,两年的项目,能学到很多东西,好像两年时间里面研究生基本都要发论文吧,适合有继续读博想法的同学。\par

另外,Emma Watson的存在也为Brown加分不少(不过她现在已经不在Brown了)。\par

总的来说Brown的CS挺不错的,花销貌似也是几个常青藤里面最低的。Brown招生的话也是很严谨的,虽然他的官网上写托福要105以上,不过有一些申请者托福没有到105,甚至没上100,也都拿到了AD,从另一个侧面可以看出,Brown招人也是和UMASS类似,并不只看申请者的硬件条件,而是要综合考虑申请者的整体素质,因此PS和CV一定要好好准备。\par

\textbf{Gatech}\par
综合排名:36 专业排名:10 申请难度:*** \par

Gatech是一个传统的功课牛校,还有个简称是GIT,和MIT CIT 并称为三大IT。有种说法是Gatech像是中国的中科大。\par

Gatech每一届招人比较多,据说有cs有200人左右吧(只是据说,没有去验证过),但是在Gatech能参与研究的机会也很多,因为院里面的教授很多,因此到Gatech读研然后再转申博士也是一个很好的选择。\par

不过Gatech招人比较奇怪,就今年的结果来看,Gatech CS找了很多转专业的学生,这一点上我不太理解,猜测是不是因为要保证学生的diversity?然后我被无情的拒绝了,其实看一亩三分地里面收到AD的申请者的背景也没有什么非常突出的地方,比较郁闷。也许是因为一眼被他看穿了我是个水货,或者嫌弃我可怜的托福口语成绩。\par

地处亚特兰大,虽然学校非常不错,但是地理位置的原因,让他的竞争力降了一个档次。亚特兰大的安全问题一直被人诟病,这也是为什么很多申请者没有申请Gatech的原因。\par

\textbf{Yale}
综合排名:3 专业排名:20 申请难度:***** \par

Yale的申请难度之所以有五颗星,并不只是因为他的综合排名非常高,而是主要因为他每年找的人特别少。就今年而言,CS master总共就发了22个AD,总共的申请人数有多少并没有说,但是这么少的AD就能体现出其申请的难度了。\par

Yale并不是因为工学院好而出名,所以人们对于去Yale读理工学并没有表示出很高的积极性。的确,Yale的理工学比起他的文科法学什么的差了一大截,不过和其他学校比起来也还是不错的,专排20可以说明一切。名校光环,加上大差不差的专业排名,Yale的CS master竞争也算是非常激烈了,可能并不亚于UCLA。\par

Yale的录取标准看上去很简单,高GPA,高GRE,高TOEFL,这三个硬指标达标了就成功了80\%了,当然,如果你本科出身特别好,也会加分不少。不过他家的master貌似不怎么注重申请者的研究背景。所以如果有好的硬件条件,完完全全可以试试申请Yale。\par

除此之外,Yale的就业情况也很好。据说Oracle招人的时候,只要是Yale的学生,直接就收了(Oracle这个公司比较奇葩,比较喜欢名校出身+高GPA),每年也有人去G家,M家,所以一年只发22个AD,可能去10多个人或者不到十个人,平均分得的资源就很多,受到的重视程度也比其他的大规模学校要高。\par

\textbf{UPenn}\par
综合排名:8 专业排名:17 申请难度:*** \par

UPenn这个学校比较特别,他分了两轮录取,第一轮截止日期在11月中旬,第二轮截止日期在3月份。这两轮是分别审的,当时我们一起申请的同学采取的政策是申第一轮,因为申了第一轮之后,如果收到Ad,有些保底学校就不用再申了,如果是个REJ,那就得把保底学校给申上。结果申请结果很悲剧,因为第一轮的时候,UPenn按照申请者的本科学校分别筛选的,我们学校的,他就只收了学分绩前两名的同学,其他人全部REJ。\par

所以,对于UPenn这种学校,同校同专业的同学就不要扎堆了,因为申请结果下来就是按照GPA给排个序,前几名的是AD,之后的都是REJ了,至少第一轮申请是这样的(第二轮貌似要水一点)。\par

UPenn以前被说成大水校,因为招人比较多,不过其实整个学校的教学质量应该还是不错的,毕业之后就业情况也很好,具体可以在一亩三分地里面的介绍Upenn的帖子里面看到。\par

\textbf{UCI}\par
综合排名:44 专业排名:28 申请难度:** \par

UCI的申请难度不大,不过在申请的时候要注意,应该申请ICS下面的一系列项目,而不是EECS下面的项目。前者偏软件一点,后者偏硬件。申请难度不大是因为他的CS系比较大,因此招生多,不过他的毕业生就业情况还是不错的,地理位置的优势为他加分不少。\par

GPA85,TOEFL100,GRE320的同学可以大胆申请。\par

\textbf{RICE}\par
综合排名:17 专业排名:20 申请难度:***\par

Rice是个南方的牛校,很不错的一个学校,学校以小而精闻名。在德州,所以生活费学费方面比较便宜(但是相比同处德州的TAMU,就贵了很多)。Rice的cs近年来有扩招的趋势,前几年的时候一届master也就只有5~10人,去年破纪录的有19个master,今年猜想估计有20~30个吧。招人少应该算是他的优势。\par

他的劣势也很明显,地处火箭城休士顿,休士顿的IT公司很少,提供的就业机会也不多,因此就业形式的话Rice的确比不过加州学校和东北部的学校。当然,如果你有实力,找工作的问题就没必要担忧了。\par

要注意一点的是,Rice的master的学位是MCS—Master of Computer Science。换句话说,这个和其他学校的MENG项目很像,但是还是有区别。像是像在他是Course Based的,就是说学生只要上课,不需要做研究,写论文。然而不一样的地方在于,其他学校的MENG一般只有9个月,而Rice的MCS是1年半的项目,时间长,意味着学的东西要多,要扎实一些(高效率牛人可以忽略这一点)。\par

另外,Rice貌似对南大一直挺有好的,每年都会给南大(泛指南大的所以专业)的同学发一定数量的AD或者offer。\par

\textbf{UC-Boulder}
综合排名:97 专业排名:39 申请难度:* \par

UC-Boulder在排名上并没有什么优势,而且还略显劣势,不过Boulder还是有理由让人爱上他的。我觉得最主要让我喜欢上这所学校的原因就是他的自然环境。这个学校处于科罗拉多州,Boulder又是一个特别休闲的城市,整个城市的社会风气特别open,当然自然风光也是特别的不错。这一点也造成了这个学校生活费和学费偏高。\par

学术上的话,具体没有怎么研究过,不过之前向学长打听过,在boulder那边找教授做研究的机会还是有的,有机会拿到RA和TA。据说就业也还不错,因为离Denver近,Denver那边的IT公司也比较多,而且我感觉貌似科罗拉多州里面,就这所学校的计算机专业还能排的上名次,所以州内竞争还算不太激烈。\par

\textbf{TAMU}
综合排名:65 专业排名:47 申请难度:** \par

TAMU的中文名叫德州农工,很多人听了这个名字,就不想申请了。其实TAMU在美国的声誉还是不错的。学校很大,大部分学生是德州本地的人。在介绍Rice的时候也提到了,德州的学校非常吸引人的一点是他们的廉价的生活和学习成本。TAMU是相当典型的例子,据学长学姐反应,在TAMU读2年master,25万RMB是完全够了的。这相比于其他学校来讲,性价比非常高。\par

但是TAMU的CS实力的确没法和前面介绍的几所学校相比,不过单看他的信价比,那是绝对的值当了。\par

\textbf{NEU} \par
综合排名:56 专业排名:61 申请难度:- \par
NEU是大部分申请人的万年保底校,因为这个学校申请难度不大,(因此我用-表示),而且这个学校有个比较有名的地方,就是co-op,这是只在上学期间可以去公司实习,这种实习是可以算学分的,所以换句话说你可以在上学期间就能实习了,而且还能挣钱,听上去非常不错。不过,这样也是有代价的,比如你想实习多一点时间,那就意味着你要延后毕业,延后毕业就意味着多交学费,所以算来算去也不是非常的合算。\par
\clearpage
