\section{李皓寰 UCSD}
我的出国申请也算告一段落了, 回想申请时的各种经历, 教训, 感想, 发现有很多值得分享的地方, 这里写出来当时为将来申请的学弟学妹指指路吧。\par

       首先说一下我的申请结果吗, 我全申的MS: 
\begin{itemize}
\item        AD: Upenn, Wisc, UC-Davis, UCSD(Accept)
\item        Rej: PSU, UIUC, UT-Austin, UCSB
\item       Pending: Brown, UNC, Umich, Rice, UCI, TAMU
\end{itemize}


\subsection{为什么要出国?}

出国是一项重要的决定, 你要么得奉献出五六年的青春, 要么就得付出高额的花费。出国的原因可以有很多, 西方的教育, 异国的风情, 不一样的学术科研环境, 等等。 我当初选择出国,纯粹是一种随波逐流。 记得是大二寒假的高中同学聚会, 很多人在讨论GRE,TOEFL之类的东西,听着不明觉厉,加上自己当时成绩也不错, 英语基础也算扎实, 于是抱着好奇的心态萌生了出国的想法。但是当我掂量了一下红宝书的厚度后,突然就心里虚了,有种还没迈出一步就栽了个跟头的感觉。当时寒假买下红宝书时计划在接下来的大二下学期把它啃完, 而实际上直到学期结束,我才啃完三分之一。\par

这个时候我内心里出国的想法已经动摇了,单词背不下来,出国花费又那么高,读博又那么苦那么久,再加上我们这个专业不管是读研还是工作都可以有很可观的收入,何必要为难自己去经历出国申请的那些苦呢? 我一度把背单词的计划搁置了两三个月,直到有一天我看到了肖复兴写的一篇散文"年轻时就应该去远方”。我现在都觉得有些不可思议,但是回忆起来确实就是这篇文章让我的想法产生了180度大转弯。 这篇文章的一些观点确实是发人深思的,漂泊他乡,是一种苦,一种累,但更是一种挑战,一种宝贵的人生经历。 出国留学为了什么? 这个问题事实上是把出国留学手段化了。在我看来,出国留学本身也可以是一种目的,漂泊国外的种种经历和考验所带来的内在价值,远比一个海外文凭所带来的经济和就业效益这样的外在价值大得多。

\subsection{Phd?Professional MS?Acadmic MS?}

在决定出国之后, 接下来的就是学位选择的问题。 学位总体上两种: Phd和MS, 但是计算机专业(不知道其他专业有没有)里面MS又可以分为Professional MS和Academic MS. 前者适合将来想出去工作的, 主要就是上课做作业。 后者适合将来转Phd的, 除了上课外可能还会做研究, 当然也有机会拿RA。\par

鉴于读MS高额的花费, 我曾经也有个读Phd的想法,我也花了不少时间去了解Phd这个学位。在我看来,与其说Phd是一种学位,倒不如说它是一种职业。学生的本职工作就是上课考试做作业,完成这三项任务就算OK了。但是Phd不是,Phd虽然也需要上课,但这已经不是他们的重心。他们的主要工作是做研究,看论文,写报告。Phd免学费且有钱这确实很吸引人,但这实际上都是通过辛苦的搬砖换来的。虽然很多学校都有全奖(Fellowship)这种东西,但不是所有Phd都拿得到,即便能拿到,最多也就维持一年。大部分Phd都要依靠研究项目的Funding养活自己。在我心目中,Phd是一个神圣而严肃的东西,不能随随便便说你想读就去读。Phd是一群某个领域的狂热爱好者,它们愿意奉献自己的青春年华,甚至一辈子给这个领域的研究。科研就是他们的爱好,一连串的挑战带给他们的不是畏惧而是兴奋。虽然他们面对着这个世界上最强的智商考验,但是他们不求太大的经济回报,因为征服这一连串考验,就是对他们最好的回报。我觉得如果不满足这些特点,最好不要尝试去读Phd。也许我把Phd有些妖魔化了,但是我觉得妖魔化点也好,因为Phd是一个求精不求多的东西。\par

我很早就断去读Phd的念头,但是我在Professional MS和Academic MS之间踌躇了很长一阵子。这两种硕士可以类比为国内的专硕和学硕。美国的MS大部分都是前者,我所知道的的提供后者的学校有UIUC, Wisc, Rice, UNC。虽然我将来是想工作的,但是我一开始的想法是读Academic MS,  因为我害怕Professional MS的课会像大软院的课一样。我当时认为Academic MS可以在课余时间可以跟教授做项目,这样既上了课又捞了研究经历,一举两得。但是我发现这种如意算盘在课业繁重的美国大学里是行不通的。我有一个同学在RPI读的本科,他常常会做作业做到3点睡觉,如果你常常逛一些出国论坛,会发现大多数留学生都常常会有一波due赶得死去火来的经历。Academic MS实际上是一种Phd后备军,他的生活方式是与Phd比较类似。如果不想Phd,最好还是不要选择Academic MS的。我申请的时候申了3个Academic MS,申完实际上就有些后悔了。\par

总之,美国学校里不管是上课还是做研究,都是很专业的,最好不要有两者兼得的想法,你只拥有做好一件事的时间和精力。在中国可以看到不少教授旗下挂着一堆项目,这在美国是不现实的。

\subsection{GRE TOEFL}

 GT在申请中重要性可以用“鸡肋”这个词来形容: 你说它不重要,但低GT有时就会卡死你; 你说它重要,但光靠GT又很难让你脱颖而出。在这个GT通胀的时代,G1350+4.0,T105(22)已经是个很平庸的水准了,我的GT甚至还不及平庸。但是如果能有个G1500+ T110+ 那就另当别论。\par

我准备GT的过程是很不专业的,首先是没上过新东方,其次准备时间也较短。我考GRE的经历还是算幸运,当时扭转心态决定出国闯荡的时候,距考试已经只有不到两个月的时间了。那时红宝书第一遍都没看完,看过的都忘得差不多了,这个时候显然常规的准备方法已经行不通,所以我只能尝试投机取巧的办法。首先我放弃了红宝书转而看高频词汇,当时我看的那本书是新东方的”高频词汇句子填空“,好像有一本类似的书叫”要你命3000“,在时间不足的情况下,看高频词汇也算得上一种合理的方法吧。填空觉得完全拼的是词汇量,5个选项全能看懂那这个题基本拿下了,要是一个都看不懂那就只能试人品了。填空这部分的实力是需要长时间积累才能获得的,如果时间不够,建议不要在这上面耗太久。对于阅读,我当时的准备材料就是一个网上下的一组练习题,名字已经不记得了。我当时就是每天练两篇。我想强调的是阅读一定要上机练,这样可以方便你适应机考的环境。对于作文,我就花的时间就更少了,我在考前两个星期才开始学写Argument,考前3天才开始练习Issue, 要不是考试的时候Issue题目正好是我之前构思过的,恐怕我的作文就完全跪了。 GRE作文的特点就是强调结构和逻辑,观点鲜明,逻辑通顺放第一位,语句优美流畅等等都在其次。它考察的是逻辑分析能力而非语言能力。我最后GRE考了个154 167 4.0, 不到两个月内能游这个成果我已经很欣慰了。\par

至于TOEFL, 我觉得最好的练习材料就是官方指南和TPO。如果你不是一个很有口才的人,那口语一定要尽早开始练,否则到考场上时会发现自己的舌头像打了结一样。练口语也不仅仅是光说说就可以的,需要录下来,自己听听然后给别人听听,看有什么地方可以改进。给别人听是很必要的,当时我准备二战的时候狂练口语,但是却忘了给别人听,结果考的时候自我感觉良好,出分发现还只有20,复议才给了22.\par

\subsection{选校}

当你考好GT, 弄好成绩单之后, 就可以开始选校了,当然也可以在选校的同时准备PS和CV。选校也是一件很麻烦的事,我当时选校的时候完全没有头绪,于是就去一亩三分地论坛发了一个定位帖。这里顺便说一下,一亩三分地是一个很好的留学交流社区,它有很好的板块分类,并且这里面几乎没有灌水帖,这也是我最喜欢的一点。在这上面有很多人都发过定位帖,一般也都会有人回复。回复你的不一定是牛人,也许他也是个学校的小白,但是当我看到很多人都在和我同行,向着同一个目标奋斗时,我感到很欣慰。一群人一起探索未知总比一个人探索未知好,虽然交流会浪费一些时间,但至少你不会觉得孤独,这不是最有效的方式,但却是最开心的方式。以我看过这么多定位帖的检验,发现选校的档次大抵是和GPA成正比的,所以在大学前三年一定要好好上课。一般人选校的时候都会按照US News专排分三个档: 冲刺档, 主申档, 保底档。 当然不同人也会有不同的地区偏好, 想过得舒服的一般都会往加州跑, 将来想去银行的一般会去东海岸, 喜欢到处逛的一般会选择LA, NY这样的大城市, 喜欢田园风景想远离城市喧嚣的一般都去德州大农村。 我觉得选校这个事情不必太纠结,想申就申, 等到拿了AD之后再纠结也不迟。 

\subsection{PS, CV}

PS和CV是留学文书中最重要的两项, 也会最让人头疼的两项。PS因为文字更多, 所以相对更难。这里我先分享写PS, CV的一亩三分地精华帖: PS就是最好的套磁,如何用CV打开申请之门。这两个帖来自于同一个作者, 其中很多观点都很有指导意义。作者在我看来也是一个非常牛B的人物,可以看一下他的申请历程,当看到他做过的项目时,我就已经完全吓尿了。 我觉得PS要达到3个特点: 真实性, 独特性和简洁性。真实性无用多说, 独特性是很重要的, 如果你没有亮瞎眼的硬件,那就得靠独特的PS来吸引commitee的眼球。好的PS至少得要个独特的开头,我不提倡用" My name is.. ", "I am .." 这样的句子来起头,如果让committee看到这样的开头, 他们很可能就会认为又是一篇平庸的PS然后直接将后文一扫而过了。简洁性也很重要, PS一般不要超过两页, 冗长的PS和平庸的PS一样让人反感。写PS是一个痛苦的过程, 要做好改的准备,我写的PS至少大改了4次,小改就数不清了。PS最好给学长或老师看一下, 他们以旁观者的视角很可能提出一些很有用的意见。CV可以看做是一个大学经历的精炼总结,你可以写上自己做过的项目,参与过的实习,或者其他非学术经历,总之就是起一个对PS的补充作用,让委员会更全面地了解你,所以CV和PS尽量不要有重复的东西。CV在内容上也可以独特一点,但是格式上一定要专业,网上有很多CV的模板,都可以照搬。\par

留学文书中还有一项是推荐信。有一句话是: 中国人的推荐信只要不是牛推,就都一样。我一个学长还告诉过我一句话: 在南大,除了周志华的推荐信,都不是牛推(当然仅限CS专业)。 推荐信对于国外的学生可能有些分量,但是对于中国学生来说,已经可以用鸡肋来形容了。据说在美国,如果只是课程教师,推荐信不会超过5句话,所以老外一看到中国学生的推荐信就知道是学生自己写的。 这里我提个醒,如果要找老师写推荐信,一定早点说,有些很忙的老师你跟他说得晚,他就不高兴帮你写了,我就遇到了这个情况。
\subsection{申请焦虑症}

申请的时候都会有些焦虑,特别是当周围保研工作的同学都早早确定下来,各种happy,而自己却在为申请弄得焦头烂额的时候。这个时候人会心里发虚,填网申表格的时候总会检查一遍又一遍。看到论坛上的各种牛人,便会怀疑自己的实力,心想自己是不是定位过高。这种焦虑是申请时必定要经历的一个过程,我也经历过,也被它折磨过。对付焦虑的最好方法就是心无旁骛,该怎么申就怎么申,尽早申完。申完之后就大可以把申请的事丢到一边,干点别的事,静等offer/AD。愁是没有用的,没有offer/AD是愁来的。虽然网申和文书准备是一个重要的过程,但它并不能起到决定性的作用。申请时的竞争力大部分是由前三年的努力决定的,在开始准备网申的时候,你的申请竞争力你已经基本定型了。那些文书起到的作用只是让你的能力特点更好地表现出来。 所以申请到了这个阶段时,你完全淡定下来。\par

 以上都是本人关于出国申请的一些观点看法,不求正确,但求有所用。出国留学是一个很好的人生经历,如果有这个想法,不要轻言放弃。最后引用电影《当幸福来敲门》中的一句台词: "You have a dream, you gotto protect it."    