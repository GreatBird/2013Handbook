\section{姚佳玮 Standford}
随着6号接到1AD 1Rej,申请季也就这么结束了。最终7AD 3Rej 1Withdraw 0Pending,对于一个从头到尾DIY的人来说,这个结果应该很不错了。写篇总结记录下我的心路历程,也给后来人分享点不靠谱的经验。\par

背景:
\begin{itemize}
\item GPA: SE@NJU, 90.0, 2/185
\item GRE: 158+169+4
\item TOEFL: 113 (23s)
\item Paper: 国内水文一篇
\end{itemize}
\par
申请结果(都是CS MS项目):
\begin{itemize}
\item Accepted: Stanford
\item AD: Stanford, UTAustin, UMich, UCSD, UPenn, Brown, UCI
\item Rej: UCB, UIUC, Dartmouth
\item Withdraw: UCD
\end{itemize}
\par
\subsection{为什么出国}
其实这个问题我目前还没有非常确定的答案,我不像一些朋友那样,大一大二的时候就坚定了出国的志向,渐渐开始翻烂红宝书。当时我发现身边一波走得比 较近的同学都开始准备出国了,又看看自己条件好像还不错,于是就“随波逐流”,打算走一步算一步。大致的过程是大二暑假(2011年7月)上了新东方 GRE班,12月考了G,2012年5月考了T,10月份开始准备申请。这个过程中,我也时不时会思考为什么要出去,一直没有找到一个特别强烈的理由能说 服自己。\par

因为专业是工科,我刚进大学的时候大致的打算是毕业后就工作,因为与其读研,工作无论是经济上还是技术上对自己的提升都更大。尤其是到了高年级之 后,我更加意识到这几年因为iOS引发的移动互联网大潮,是继2000年dotcom和2005年SNS之后,IT界新的一轮高潮。我的实习老板也说,现在国内的创业环境并不比国外差,而且有更加大、更有潜力的市场,留在国内可以是更好的选择。所以对于那些技术强,又有想法的同学,我觉得拿了文凭赶紧工作或者创业才是正道(你甚至不用毕业就开始创业,如今开发测试发布一个产品的成本非常低,我知道已经有学弟在这么做了)。\par

我实习的单位是一家创业公司,大小老板都曾供职于大型上市公司,也都各自创业成功过,可以说是非常靠谱的。他们和我(还有公司里其他要出国的实习 生)都谈过很多次,聊美国留学和工作的情况,以及公司的前景,全都合情合理,也让我更加动摇。想想作为初创员工,如果过几年公司上市了,我就啥都不用愁 了。不过我也发现,即使他们说得很在理,也有种难以言喻的力量拉住了我——有个说法,如果在两个选择间摇摆不定,就抛硬币,如果抛到A面却想要再抛一次,答案就不言自明了。\par

罗嗦了这么多,我对自己为什么出国的总结如下:第一,趁年轻还有理想,出去看看,丰富下人生的阅历;第二,我向往国外的环境,同时不太喜欢国内的环 境,出去看看究竟哪个更适合自己。第三,IT说到底还是国外领先,我的目标就是去美国工作。(所以除了第三天都是不靠谱的理由。)\par

\subsection{软硬件}

软硬件的问题我觉得可以用一句话不靠谱地总结:硬件决定申请的下限,而软件决定其上限。\par

众所周知,有的学校是GPA控,还有的卡GT,如果你的dream school正好是其中一种,而你的硬件又不达标,肿么办?所以,硬件越高越好,就是这样。具体来说,刷GPA的话,我们学校还是很好的,据说重修了好像 不会留下痕迹;GT呢,多半实力小半运气,准备充分再考,争取不要二战。另外,由于大多数MS项目对科研的要求不高,准备申MS的同学硬件尤其要过硬。准 备申PhD的同学呢,赶紧写你的论文去!\par

科研的话,对申PhD的同学来说是必须要有的,paper几乎是必须的,没有的话(听说)会遭遇各种冷眼。对于MS来说,有了肯定是加分的(好像我们申MS里面有paper的结果都蛮好的),尤其是在你有可能读完MS继续PhD的情况下。南大的学弟学妹们,我强烈建议你们参加我们学校的本科创新训练计划,据说最终能发论文的人不少,就算没有,也可以在CV添一条科研经历是吧。我当时就是参加了这个,最后混了篇国内会议的paper。另外如果大家对软件测试或者推荐系统有兴趣,可以找陈振宇老师当指导老师,陈老师不仅学术过硬而且为人非常nice。\par

文书的话,CV是必须的,PS、SoP一般学校两选一,也有的学校都要。我个人的感觉是(如果一所学校都要的话),PS偏重个人的经历和体验(也有 学校叫PHS, Personal History Statement),而SoP则偏重学术和科研,换句话说说SoP比PS更加正式一些。不过实际过程中,有的学校要求的PS实际是SoP,有的则相反, 或者干脆是mixed style,所以还是按说明,不要想一招鲜吃遍天。提醒下,这里的要求是指学院Admission Requirement上的要求,而不是申请系统里写的要求,因为后者绝大多数情况下是整个graduate school通用的。我的建议是PS写一份,SoP写一份,或者写PS和SoP的混合版一份。\par

我觉得文书除了必须要真实外,特别重要的一点是要呼应,所以我比较推荐先写CV,这是你到申请为止整个大学的总 结,然后选取其中的highlight写到PS里,这样就做到了呼应。另外文书和推荐信也可以呼应的,比较理想的情况是,三封推荐信分别来自任课老师、科 研导师和实习老板,这样就从学习、科研和职业三个方面证明你的能力,也让你的PS更加有说服力。\par

当然实际情况你可能会发现没什么好写的,这个很正常,你要做的就是发掘自己的亮点,你是不是学习很用功(废话)?在某一门专业课上有没有突出的表 现?参加学科竞赛有没有什么achievement?在学生会是不是体现了leadership?对自己的专业有没有一些独特的想法?PS不是一次搞定 的,随着你对自己了解的深入,你会觉得思路越来越开阔。\par

我不建议大家套PS模板(SoP的话还行)。上面说了,文书作为软件,是提升申请的X因素,如果你的文书过于 standard,对committee来说就和硬件无异了,甚至会给人非常没有诚意的感觉。最理想的情况是,如果你精力足够,针对school的特点和 自己的match程度都单独写文书。我的话,因为拖延症,PS和SoP各写了一份,给Stanford的SoP单独写了一份(下面细说)。\par

CV的话就不要独具匠心了,按照基本的模板写好,但注意字体不要过小,一定要方便看。CV作为你目前为止整个career的summary,直接体现了你的competitive程度。传说committe申请人数太多的时候,一般会先看CV,PS次之。所以硬件之后,CV或许是你的第一道防线,不要给看得人surprise。\par

最后提一点,会用LaTeX的同学一定要用LaTeX,只会用Word的同学希望你花时间学一下LaTeX,这 样突出你的专业性。众所周知committee一般是教授组成的,不排除有些教授看不惯Word排版出来的文档。(这里不是说Word排版不如 LaTeX,而是说一般人Word排版结果远不如LaTeX默认设置)。另外用Mac的同学可以用XeTeX,可以直接调用各种字体,很方便。我CV、 PS用的字体是Palatino,我觉得比Word默认的Arial要美观和正式。\par

\subsection{选校和定位}

其实这点我也没什么值得说的经验,套用一位学姐的话就是“美国的Master项目都差不多”,所以选校不像PhD申请那么细,需 要精确到研究领域和教授。我的方法是,从一亩三分地上挑往年和自己背景(学校、软硬件等)差不多的,参考他们的选校列表和申请结果,加上一些自己比较有好 感的学校初定一份列表,然后去这些学校的系主页看硬件要求,学费、地理位置、FAQ等,从中筛选出一份选校列表。我觉得大多数人除了对几所(比如 dream school)特别了解之外,其他学校其实没有特别的感觉吧,所以这个方法应该还是比较有效的。\par

我想特别说的是,不要去特别在意“水校”、“申爆”等留言甚至各种阴谋论,你要相信每个学校对申请者的评估过程都是公平的,就像committee 会相信你的材料都反应你的能力一样。而且那些被申爆的学校肯定还是被申爆,同时会导致一些本来不被申爆的学校也被申爆,所以MS的趋势就是申爆,不要管这么多。\par

另外我建议大家关注下rolling的学校(UPenn,UTD,NEU等),所谓rolling就是先到先处理,这种学校非常方便结束三无状态。比如UPenn有三个cycle,第一个deadline在11月15号,11月底就出结果。如果拿到AD就是吃定心丸了,就算拿到Rej也可以拿来校准你的定位。\par

地理位置的话,就CS而言,加州和东北我觉得都蛮不错的,旧金山和纽约两大创业地带嘛。就我而言,加州的IT企业,尤其是湾区附近更多一点,加上加 州的气候听说也很好(据实习小老板说,波士顿一年5个月是暴风雪,可能夸张了:-P),所以加州对我的吸引力更加大一点,最后11所里面有5所是加州的。\par

\subsection{网申}

先说寄材料吧,我申的学校只要寄GT和成绩单,都比较顺利。G需要在ETS官网用信用卡付款,如果实在不行可以试试Paypal。T则方便很多,在 etest上就搞定了,不过ETS系统上可以知道成绩送出了没有,而etest上没有这个。成绩单的话我都是用EMS和同学合寄的,据鼓楼邮局的大姐姐 说,EMS到了美国走的是美国邮政的快递,就我的经验来看速度还是可以的,一周可以到。不过我有同学遇到了EMS投递失败的情况,所以图保险的话就用联邦 快递或者DHL吧。有些学校明确要求材料必须在网申deadline前到,加上邮政系统和ETS的不确定性,材料早送才是王道。\par

网申基本上是体力活,填多了——尤其是一些奇葩系统填多了——也就习惯了。我申的11所学校网申系统分为了3类,Stanford, UMich, Brown用的是ApplyWeb(也叫CollegeNET)的系统,UPenn, UIUC, Dartmouth, UCD用的是ApplyYourself的系统,其他学校是自己的系统。ApplyWeb最大的特点是只能在section之间prev和next,如果 一次申请你想要多次填完就会很蛋疼,相反如果你从头到尾一次搞定就会很顺手。ApplyYourself系统择相对人性化很多,可以在section之间 跳转,但是从技术上来说不及ApplyYourself。学校自行开发的网申系统易用程度就参差不齐了,我用下来UCB和UCSD是所有系统里最好用的, 而UCI的系统明显很弱。\par

网申完毕并不代表申请就好了,因为你添得内容和之前寄的材料还需要match起来才算是一份正式的application,好像大多数学校都是等材料全了才开始review的。所以网申完多去学校提供的tracker上查状态,如果很久都没match到一定要联系小蜜。\par

\subsection{关于结果}

结果无非就是Admission或者Rejection,前者让人开心后者让人伤心。还有一种特殊情况,如果有人AD有人Rej你却没消息,那你就被WaitingList了,等拿到AD的人decline了才会轮到你。\par

就今年的经验看,CS MS集中在三月中下旬出结果(PhD好像一月就开始陆陆续续出了)。我除了UPenn的AD是11月底收到的,之后一个AD是2月底到的,最晚的一个是前几天(4月6号)刚收到的。所以就算连续Rej也别太灰心,战线很长呢。\par

MS项目的话,大多数学校都是遵守4/15的,就是学校不能强制学生在4月15日之前答复,即使accept了也可以反悔。所以4/15之前,如果 你有多个AD,不要提前纠结,如果来了个dream school之前的纠结就白费了是吧?当然一般情况是,有了几个可以take seriously的AD,这时候除了纠结,记得把来了AD也不会去的学校给withdraw掉,不仅对其他申请者有利,还能攒RP。同理,如果你最终决 定accept一所学校,其他给了AD的学校也decline掉。\par

\subsection{团体作战}

我想特别对DIY的同学说一下,申请是拉锯战,能团战就团战最好不好单干。我觉得我申请的成功很大一部分要归功于身边的一群好朋友,如果没有他们的指导和帮助,整个过程我肯定不会一帆风顺。\par

团战的好处很多,比如消息可以共享、选校在可能的情况下能避免扎堆(据说committee会进行同校比较)、文书可以互改(这点很重要,有时候文 书虐你千万遍,错误你却看不见)、成绩单可以合寄(省钱啊亲)等等。最重要的是,在DIY之路上有了一群志同道合的伙伴,你受到的挫败感会少一点点、成就 感会多一点点,朋友也会变多,说不定最后继续当同学呢。\par

\subsection{关于Stanford}

可能有人对这个感兴趣,我就弱弱地分享下我的心得吧。\par

其实能去Stanford真有点机缘巧合的味道。我在去年4、5月各种实习招聘都没过,心灰意冷的时候歪打正着进入了上面说的创业公司,从去年7月 一直干到现在。老板是某互联网公司的前高管,人很nice。12月的时候差不多实习5个月了,就哆哆嗦嗦像老板求了个推荐信,老板不仅欣然答应而且亲笔 写,我觉得最终这封非常strong的推荐信给我的application加分不少。所以经验之一:正确的选择才叫机会,但只有当你回头总结的时候才能发现。\par

上面说了,我的PS和SoP各写了一份,根据不同学校的要求裁剪,但Stanford的SoP我是用完全不同的思路写的。我是这样想的:拼硬件我肯 定拼不过其他牛校的牛人,不如从Stanford的文化切入,说不定能有亮点。整篇SoP我以startup为主线,讲创业的重要性,硅谷的创业文 化,Stanford和硅谷的联系,以及我自己立志创业和参与创业的经历,加上老板那封非常supportive的推荐信,也算自圆其说了。虽然也有Stanford是GPA控的说法,我还是觉得Stanford这个AD还是非典型胜利,或许的确是文书推荐信起得作用更大。所以经验二:文书要有针对性,还要有推荐信back up。\par

其实CS四大里面除了CMU都是我的dream school,由于MIT没有单独的Master项目所以没申,当时申Stanford和UCB都是冲dream rej去的。3月23日我收到UCB的拒信,虽然知道自己的确不够格还是很伤心,郁闷了两天。结果26号一大早收到Stanford AD,到现在还觉得特别不真实。经验之三:申请大起大落太快,要有一颗big heart。\par

现在想起准备GRE那段苦逼的日子,想起磨蹭论文那段煎熬的日子,想起定位选校毫无头绪的日子,想起被网申折磨的日子,想起收到AD时的快乐和收到Rej时的伤心,每一个画面都像是刻在了记忆的石板上,宣告着这个结果的来之不易。唉,开始矫情了。\par

事前猪一样,事后诸葛亮,这就是我的申请总结。\par


版权申明: 署名-非商业性使用-禁止演绎 3.0 中国大陆 (CC BY-NC-ND 3.0 CN)