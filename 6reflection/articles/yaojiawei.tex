\section{姚佳玮}

背景:
\begin{itemize}
\item GPA: SE@NJU, 90.0, 2/185
\item GRE: 158+169+4
\item TOEFL: 113 (23s)
\item Paper: 学校创新项目,国内水文一篇
\end{itemize}
\par
申请结果(CS MS):
\begin{itemize}
\item AD: Stanford, UTAustin, UMich, UCSD, UPenn, Brown, UCI
\item Rej: UCB, UIUC, Dartmouth
\item Withdraw: UCD
\end{itemize}
\par
\subsection{关于定位和选校}

MS的申请没PhD那么有玄机,基本是拼硬件。定位全看背景,学校、GPA、排名、GT、科研等都是要考虑的因素,一亩三分地W大有免费定位,认真填表即可,他会给你一个大致的方向,参考价值较高。偷懒一点可从往年的申请者中找和自己背景差不多的参考,这在没有太多精力了解每个学校的情况下很有用(说说而已啦,多花点时间啊少年)。\par

选校的话记得找自己顺眼的,不要最后给了你AD都不想去。通过人家主页多了解下,不明白的地方尽情问小蜜,不要钱的哦。数量10-20左右,记得拉开档次。还有不必迷信“申请的人太多肯定当炮灰”、“这个学校不收我们学校的人”这种说法,出国的越来越多,每个学校申请人数增加是肯定的,另外每个学校给AD的标准都不同,不要自己给自己限制。\par

申请材料和文书的具体指导可以参考我的这个\href{http://d.pr/f/UgYx}{slides} 。

\subsection{关于申请}

大家可以多关注下rolling的学校(UPenn,UTD,NEU等),早提交早出结果,方便结束三无状态。比如UPenn有三个cycle,第一个deadline在11月15号,11月底就出结果。如果拿到AD就是吃定心丸了,就算拿到Rej也可以拿来校准你的定位。\par

排名靠前学校的deadline扎堆都在12月15日,由于申请过程中不可控的因素太多,大家就不要玩火踩deadline了,会吃不好睡不香(比如像我一样T\_T)。还有到申请末期免不了手熟就不认真看instruction,如果这所学校不是可有可无,还是认真填吧,不要出岔子了(比如我申UCD的时候手一抖把出生地填成Central Repub of Africa)。\par

网申完毕并不代表申请就完了,勤查状态,要等材料、GT成绩都到了才能放心。之后就等结果吧。

\subsection{关于申请结果}

就今年的经验看,CS MS集中在三月中下旬出结果(PhD好像一月就开始陆陆续续出了)。我除了UPenn的AD是11月底收到的,之后一个AD是2月底到的,最晚的一个是4月6号刚收到的。更有同学5、6月还收到AD,所以就算连续Rej也别太灰心,战线很长呢。\par

MS项目的话,大多数学校都是遵守4/15的,就是学校不能强制学生在4月15日之前答复,即使accept了也可以反悔。所以4/15之前,如果你有多个AD,不要提前纠结,如果来了个dream school之前的纠结就白费了是吧?当然一般情况是,有了几个可以take seriously的AD,这时候除了纠结,记得把来了AD也不会去的学校给withdraw掉,不仅对其他申请者有利,还能攒RP。同理,如果你最终决定accept一所学校,其他给了AD的学校也decline掉。

\subsection{关于Stanford}
之所以Stanford收了我,我觉得主要有三点原因:\par
\begin{enumerate}
\item 硬件足够。对于MIT、Stanford这种学校,我们学校肯定不如清北有优势(人家排名20、30的都非常有竞争力),所以top的GPA和GT成绩是不被刷掉的保障。这里说一下,S的T要求113,这个和布朗的105一样,其实不卡,不过入学后可能要上英语课。
\item 有分量的推荐信。给S的三封推荐信各自来自课程老师、科研导师和实习老板,内容都比较strong,从学习、科研和工作三个方面证明我的能力,算是比较让人信服吧。
\item 有亮点的文书。S的SOP我是单独写的,整篇以startup为主线,讲创业的重要性,以及我自己立志创业和参与创业的经历,算是和Stanford代表的硅谷创业文化比较match。加上之前是互联网上市公司高管、现在在创业的老板的推荐信,也增加了说服力。
\end{enumerate}

\subsection{一些建议}

出国申请总体上来说是一场拉锯战,希望大家能尽早、认真准备。大一大二好好刷GPA,大二大三好好考GT,大三大四扎实文书和申请。知己知彼,百战不殆,最了解你的就是你自己,所以我不建议把一切都委托给中介;有时间就多了解了解目标学校的招生特点。多上一亩三分地,多和一起申请的同学交流,stay tuned,不要孤军奋战。\par

另外,出国交流是提升背景的有效手段。身边无数的例子表明,去过A校交流拿A校保底的可能性就很大,就算不能保底,能拿到老外的推荐信也更有分量。\par

人人上有一个我申请的总结,有兴趣的戳\href{http://blog.renren.com/blog/282030513/900352093}{这里}。

祝大家都能申请到理想的学校。我的邮箱:kavinyao@gmail.com