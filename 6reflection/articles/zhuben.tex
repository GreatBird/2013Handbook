\section{祝犇}
背景:
\begin{itemize}
\item GRE: V 155 Q 170 AW 4
\item TOEFL: 108 R29 L28 S23 W28
\item GPA: 85.4 RANKING: 16/185
\end{itemize}\par
申请结果:
\begin{itemize}
\item AD: SIT, WPI, UBC, TAMU, McGill, UCI, RPI, NYU, Cornell
\item PhD Offer: CWRU
\item Reject: UPenn, UTAustin, Toronto, GT, JHU, UCSD
\item Pending: Alberta, Waterloo, McMaster, Brown, OSU, VT, Brandeis, CWRU, USC
\end{itemize}
\subsection{关于出国}
本人很懒。。。从小到大一直比较贪玩吧。。。在学校里一直混啊混的,一心要安逸的会小日子。。。也栽过不少跟头。。。不过目前看来结果还不算差 o(╯□╰)o\par

我是高中的时候就想着大学毕业之后出国的,高三暑假年龄差一点没有去学车,老爹怕我太闲就让我去新东方上托福,虽然当时没报名考试没好好做练习,但还是或多或少充实了大学期间被各种啃的老本了的(天天空调下坐太久上课最后落得腰疼尾椎疼了好久。。。怨念)。\par

大一进学校还带着了本托福词汇,英语分级考试前还装模作样的翻了翻当复习,不过之后么就只能呵呵了。宿舍里有俩转系的大神,不用军训还有电脑,所以从开学我们宿舍的游戏氛围就灰常好,自己买了电脑之后,大学里没人管了,顿时省略一万字,大家懂的。报应来的总是很快的,大一下三门数学,高数华丽丽滴挂了=。= 万幸有暑期重修班给补救回来了 终于意识到大学可不是那么好混的啊啊啊 不过也因为高数挂了更坚定了出国的想法(当年可是说挂科不能保研的- -)\par

大二开始了好好混日子的节奏,趁着课不多还顺手拿了个驾照。大三按照出国的目标开始按部就班的准备G/T什么的,木有研究经历就暑假实习算点实践经验,大四就开始申请各种工作,然后就到现在了 =。=

\subsection{关于GPA}
对于Master的申请,我觉着最重要最基本的就是GPA了,Phd的话可能research更重要的。首先GPA这个是一定要的,低GPA申到好学校的不是没有,但要有其他方面特别突出,而且毕竟是个别不具有参考价值。要有比较好的GPA,就要争取每门课都有不错的成绩,核心课必须要努力,有些学校在看你的成绩的时候会着重看核心课;选修课在满足自己方向和兴趣的前提下,可以根据学长学姐的经验酌情选一些可能比较容易高分的课规避一些比较坑的课。\par

如果有关键的课成绩不好,推荐去重修,能刷就刷,但是重修不是最后去随意考个试就好的,必须要认真对待,不然刷低了不要说我没提醒过。选修课是可以注销成绩的,所以可以多选几门,之后把差的注销掉。不过要注意的是,学分必须要满足毕业要求,而且每个学期只能申请注销2门课,所以如果有选修课成绩比较差不想要了要尽早注销,不然就只能揣着了。最后吐槽下软院,我们的GPA拿出去和其他很多学校比实在是太吃亏了,学院应该多关注关注自己的学生啊喂,不能给学生的前途制造障碍呢。

\subsection{关于GT}
GRE其实没什么多说的,单词是关键,只要结结实实的把单词背好,加上适当的应试训练不会差的。背单词不是说每次就看一遍什么的就行了,要把中英文意思、常用词组、用法什么的都要记住,不然像我这样的考试的时候做填空看到单词都是这个词我背过的具体的不记得了,就妥妥的跪了=。=\par

TOEFL推荐在GRE之后考,一是有效期要短,二是考过G再考T你会发现单词、阅读和写作都是小菜一碟了,听说需要比较长时间的训练,毕竟T本身就是考查的英语本身的能力。不过如果你跟我一样花10天时间准备T,时间不充足,那么可以适当用点应试技巧,听力阅读抓住出题的套路,口语和写作准备好一份不错的模板,也是能过关的。不过不推荐啦,努力付出认真准备才是硬道理。

\subsection{选校和申请}
首先我推荐申请要DIY,只有自己才是最了解自己的,就算要找中介,文书也要自己写,不然实在不靠谱,而且通过申请也能让你更清楚的认识自己。\par

申请是个大工程,需要大量的时间和精力,所以我认为越早准备越好,GT除外,需要你去看学校,了解学校和项目信息,准备各种文书。我因为申请的学校太多,从10月底开始网申,到12月初才算是全部搞定。所以如果你能早点开始申请工作,11月完成申请是比较理想的。早点申请材料就会比较早得到处理,即使出了什么问题也能比较从容的补就,申请这么重要的事情最好还是不要和作业一样不到deadline不启动了=。=\par

选校的话,要有层次,比如分冲击的梦校、主申的和保底的,每个层次选几个学校。以我的经验来看,可以多申一些,尤其是申ms,不要怕多不要怕烦,拿了一把AD挑一个总归是比没得选要好的多的。\par

这里我还要着重推荐一下加拿大的学校,很多人都是申的美国,其实加拿大有不少学校还是很好的,而且大部分学校对硕士都提供资助,这一点就是很诱人的嘿嘿。

\subsection{关于文书}
在GPA和GT一定,不易提高的时候,你的文书就是申请工作中最大的变量了,努力提高你的文书质量才能让你的申请更出彩。\par

两点比较关键的,一是自己写,不要照模板不要看别人的,一定要自己写,写自己,写写你的本科学习、研究、实习经历,对你申请的专业和方向的认识,为什么要出国为什么选这个学校等等;二是多修改,一定要多修改,开始的时候哪怕推到重写也不要怕,一定要注意文书的逻辑关系,要连贯要能让人信服。语法用词什么的可以找native speaker或者一些文书机构来修改,这个也是很重要,我因为老爹是英语老师所以果断坑爹去了,所以你也可以尝试找英语老师、外教什么的帮忙。

\subsection{关于软实力}
研究、实习、交换经历,这些不是必要的,但是如果有,不仅对你的申请,对你个人的成长也是很有帮助的,所以如果有机会的话,果断勇敢的去吧少年。\par

研究经历的话,可以参加院里老师的研讨班,跟着做项目或者做自己喜好的方向,做的好如果能发paper就大神啦;个人觉得研究经历还是满重要的,phd和加拿大学校的面试就跪了啊- -\par

实习的话大三下就会有校招了,暑假就可以开始实习。\par

交换的话,说实在的因为软院的三学期制在学分转换上会比较蛋疼,大二或者大四上是比较理想的时间,如果有暑期的高质量项目也可以考虑。\par

最后祝愿出国不出国的同学都能有理想的结果,准备出国的学弟学妹们也能早作准备有理想的申请,如果有问题也可以联系我,人人或邮件什么的zbtcdx@163.com
\clearpage

