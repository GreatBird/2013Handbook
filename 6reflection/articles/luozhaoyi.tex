\section{罗钊燚}
时间仓促,匆匆写下此总结,文笔略差,多多包涵。水平有限,学神请绕道。\par

本人南京大学软件学院09级本科生一枚,一共申了11所学校,其中美国8所,加拿大1所,英国2所。全部申的CS Master。\par
\begin{itemize}
\item Accepted: Waterloo CS的Master,32000加元一年(包括RA,TA)。做自动化测试、导师不错。
\item Decline:Cambridge,奖学金是offer发了后再办的(英国的奖学金不一定是学校出钱的,很多机构funding那种),decline之前没拿到奖学金。
\item Rej:Purdue、Yale、Princeton,Oxford
\item 其他:CMU CS\&EECE, Michigan, UCLA,Columbia
\end{itemize}\par
条件:GPA 4.2,有出国交流经历,会议paper一篇(IPDPS,B类偏上)。雅思7.5(美国大部分学校还是认可雅思的,一般要求7分),GRE 157+169+3.5\\(2013年1月5号),153+168+3.5(2012年11月25号) \par
关于GPA、上课:平时学习比较随意,考试压力不大,导致成绩普遍不高(大多80出头,大概有10门70多的),但学到好玩的东西就很来劲,几门核心课程成绩不错(数据结构96,机组原理96,操作系统93,网络92,编译原理89,算法88)。总体来说GPA偏低,成绩分布方差很大。大三出于装逼的目的坚持不重修、不注销、不为学分绩折腰,直接导致与不面试的美国MS program基本无缘,现在想来挺后悔。我的经验是:能考好的课程尽量一次性考好,失误了的如果在80分以下最好重修掉。\par 
\subsection{经历}
下面按我最擅长的方式(记流水账)写下经历吧:\par

大二暑假参加了为期2个月的交流项目(Pembroke-King’s Summer Program, Cambridge),选修3门课程,另外可以选择tutorship,名额有限,有Cambridge导师一对一指导做2个月的研究项目。在此隆重推荐,在各方面都能学到非常多的东西,去了绝对不会后悔(在此不详说了)。申请过程和国外Master申请的流程差不多,要求成绩单、一封推荐信、雅思7分、一份written sample等,review 3周后给答复。2个月的费用约为学费2000磅+生活费1500~2000磅,tutor额外500磅(有cover学费的奖学金,名额很少,而且要提前申)。\par

 

大三下学期,从2012年1月到10月,大部分时间在做研究,花了很大精力(不是创新项目,就是普通的跟着老师做那种),最后paper发在一个国际会议上(IPDPS,B类偏上)。经验:科研经历最重要的不是经历也不是发paper,而是提升自己的学术素养。面试的结果通常是决定性的,压倒其他一切经历、GPA、获奖证书等,而面试考察的就是学术素养。可以认为,在有面试环节的申请中(英国的牛剑、加拿大的学校以及美国的PhD),其他经历仅仅为了得到面试机会而已。\par

 
\subsection{申请}
下面是申请(我着重说一下加拿大和英国的,美国的悲剧了就不说了):\par

paper写完就是10月初了,时间已经很紧了。我花了1个月时间背GRE单词(平均每天7小时),背完后11月初颓废了10来天宅宿舍玩游戏,然后2周准备GRE、刷题(作文基本放弃),11月25号考G,接下来3天写申请文书、弄推荐、提交了5个学校的申请(马上申请就截止了),然后开始了长达几个月的颓废。期间,12月中旬申了剩下的6所学校(基本也是马上申请就截止的);套过CMU的一个教授,跟我一样的做云计算,我跟他说我看过他的paper,很有兴趣(实话),然后把我做的paper发给他,结果杳无音讯;12月12号Cambridge电话面试,面试内容是先自我介绍,东拉西扯,然后对一篇提前发给你的paper做分析(所谓的学术素养)。面试发挥得不够完美,本来感觉达不到Cambridge这种高端学校的标准,但1月17号还是收到了condition offer(莫名其妙)。12月30号Oxford一面(或者叫笔试吧),用邮件发一些算法问题,作答。回答自我感觉很满意。1月11号Waterloo的一个教授发来邮件,说看了我的简历不错想邀我面试(传说中的被磁套?),然后我跟我们院里的老师打听了下这个老师,据说挺牛。面试的电话打了大概半个小时,仍然是先介绍下我大学的学习情况(他会在他感兴趣的地方打断我问我问题),然后东扯西扯,扯得差不多了我说我看过他的paper(面试前几天看的),他说that’s great! 然后开始扯paper的内容(发挥所谓的学术素养的时候又到了),面试结束时他说It’s very promising, I am not decided yet but I will definitely take this seriously。然后我们通过电子邮件keep in touch,我继续看他的paper,说说自己的想法,也会回答他问一些问题。大概过招4、5回合后他就从了,决定招我。2月6号offer正式发下来了,2月10来号我accept了waterloo的offer,decline了Cambridge。2月底Oxford二面,问的是马科夫链的不可归约性、极限概率解唯一的条件(因为我发的paper里用到了马科夫链,他以为我比较了解)。结果因为已经从了waterloo的offer了,没好好准备,知识也早忘了,面得惨不忍睹(学术素养不够高啊)。不出意外面试后的第二天就受到rej了。后来我把这些基础知识再狠狠地学习了一遍。

 
\subsection{经验}
下面归纳几点重要的经验:\par

GPA别太轻视,目标是每门80分以上。80分一下的最好重修,但尽量一次性就搞定,因为重修也是要耗时间精力的。\par
出国交流重要的不是经历,而是提升综合素质。。\par
不要好高骛远。研究刚起步就把top conference如数家珍似的了解个通透是没有用的,就像没钱买一个轮胎还整天把各种豪车挂在嘴边一样可笑。脚踏实地地一心做学术才是硬道理,肚子里有货了再来了解这些也不迟。\par
搞研究重要的不是经历,也不是发paper,而是提升学术素养,面试时也用得着。\par
心态要端正,国外不是天堂,不是dream,结果怎样都无所谓。反正我是抱着收齐拒信回家种田(码农也是农)的觉悟做的申请\par

最后,我自认为对大局观把握得比较好,而不擅长各种细节(或技巧)处理吧,如文书怎么写,申请怎么弄,套磁怎么搞,单词怎么背等等没有有用的经验可以分享给大家,自己也没觉得这些做得有多出色。另外,经历较奇葩,大家挑着顺眼的看就行,希望能对各位的申请有所帮助。祝大家offer多多!