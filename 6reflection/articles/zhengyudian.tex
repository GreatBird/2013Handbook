\section{郑玉典}
得到offer的那一刻,心情是十分激动的,手在抖....缓过神来,分析下自己这四年的一些经历和尤其是这段时间辛勤的付出,希望阐述一些自己的看法,我觉得我也有义务去把自己的认为这个过程中需要focus的地方讲出来,给即将申请的,或者说是再过一两年需要申请的,或者说刚上大学的童鞋给予一些帮助吧。同时也希望大家带着批判的眼光来看,因为底下大部分是my own perception。\par

首先想先介绍一下香港的大学的情况,一般来说,香港的大学分三种学位:
\begin{enumerate}
\item MSc,这种是一年的,需要授课的,需要交学费,这种学位门槛稍微低一点,在我们学校学分绩80+的话有希望申请一下,不过以前也出过85+学分绩落败的例子,总之学分绩也不是决定性的;
\item MPhil,这种是两年的,全额奖学金,门槛比较高,因为申请这种学位的学生很多是拿香港当跳板的,所以一般只录取985和211学校排名非常靠前的学生,学分绩暴高其他能力又不是很差的那种;
\item PhD,这种是四年的,全额奖学金,考验全面素质比较强,当时记着在合肥初面在Waiting Room里面rest的时候HKU CS的Head跟我们聊天让我们列出来PhD需要的素质:originality, motivation, programming skills, hardworking, communication skills... 应该说香港的PhD强度还是蛮大的。
\end{enumerate}
\par
具体的很详细的信息大家可以经常到寄托家园港澳版去浏览一下。不过大多数学校都支持MPhil和PhD的互转。拿去年港大的总体招生情况来说(听去年过去的一个MPhil介绍的),一般MSc给80个左右,MPhil+PhD只给24、5个(其中提前批可能能发10个左右的offer)\par

一般来说CS这个系比较特殊,香港有很多学校都只在这个系设提前批(HKU,CUHK,HKUST),提前批的难度说实话是非常大的,提前批只给MPhil和PhD的Offer,今年港大走了全国五个城市,香港,合肥,北京,上海,广州,港大提前批做的非常棒,有个专门的\href{http://i.cs.hku.hk/~gradappl/index.html}{网站}大家可以看下。从另外一篇日志得到的信息(具体可以看我的分享,是一个东大兄弟讲述的我们在合肥面试的经过)是进入初面的有150人,最后给Offer仅有10-15个,竞争是非常残酷的,我不想详细地描述这个过程,我只想讲一下大学四年我是如何一步步完善自己的,以及申请的时候真正需要focus的地方,我觉得这个是最主要的。\par

如果让我用一个最关键的词语来描述我hku提前批的突出重围,我想用motivation来描述。主动性这个点太关键了,首先我们是南京大学的学生,的确占据着一些优势,像清北中科交大南大这些学校在香港的口碑都非常好,但是我学分绩并不是暴高那种,打出的成绩单上最后的学分绩也就在85+,其实申请PhD的话教授更喜欢考察你全方面的素质,你大学里参与了什么研究?你都做过哪些令你很proud的项目,你还有没有一些其他特殊的经历?因为我真正觉得学分绩只是教授参考的一个小项,很多其他的你做的东西真正highlight你的大学生活,也真正会提起教授的兴趣。\par

拿我自己来说,大学里我在简历里主要highlight的地方有:大一时候EL比赛的第一名,大二时候C++编推箱子游戏实现了在特定区域内(因为这是个NP-HARD问题)自动解推箱子的算法,大三的时候加入了周志华老师的LAMDA组;担任大二学生的OS实验课助教,讲课并写了详细的ppt,advanced materials以及home assignments;跟着coursera完成了Daphne Koller的stanford的PGM(Probablistic Graphical Models)课,这是一门非常advanced的课程。\par

我真正觉得这些经历是highlight我的地方,因为教授不会盯着你的成绩单问你哪门课为什么低,他们更感兴趣的是你如果全面地发展你自己,能不能应对非常困难的工作,是否有足够的motivation去挑战自己,因为做研究毕竟是个很challenging的工作。为此我还专门做了一个\href{http://www.zhengyudian.com}{网页}介绍自己,很简单的网页,就是介绍一下你所做的就OK了,有兴趣的可以看一下。做网页为什么重要?因为教授一般不会很详细的看CV里面大段的文字,因此你可能需要一个略微动态的东西来展示你自己。\par

下面说一下我们如何利用资源来提升自己研究的能力,提早进入实验室。对于本科生来说,我们几乎所有人没有时间去从事一些研究性质的工作,因为研究一件东西需要的东西非常多,你需要了解这个领域基础的mathematical derivations,需要知道这个课题目前做成什么样子了,需要把以前人们做出的结果发的论文认真看一下,这些需要的时间是很多的。说句实话,我觉得做研究很需要GEEK精神的,是不断探索一些事物的能力。而进入很强的实验室让我们作为一个稍微参与研究的人去真正理解研究是干什么的成了很必要的一件事情。因为DM比较火,很多领域都拿ML的technique来做。我大三上学期跟着Andrew NG在stanford的CS229:machine learning课认真看了几遍,里面讲了一些基本算法的数学推导,几乎是完全讲数学推导的,让你了解基础的算法为什么会perform 这么well,大三下学期加入了周志华老师的LAMDA组,周老师的名气就不多说了,我在港城市的一个不怎么与cs沾边的同学的导师都曾经跟他提起过Zhi-Hua Zhou,周老师是国内难得的真正而且很纯粹喜欢做ML的理论研究的人,也是难得的没有在国外读PhD回来的人,纯粹是一步步自己过来的人。他们组应该说是难得的好组,PhD和master都在一个实验室里,这样有时候在他们实验室请教问题非常方便,因为他们组PhD大都都是涉及理论研究的,很多是分析算法的,所以当你某一块不太明白的时候他们能给你讲的非常形象,这块的确是非常赞的地方。\par
      

每次与老扬交流非常好的算法的时候都能得到新的收获。DM的范围非常大,现在很多领域都拿DM的一些technique来做,而且DM是与业界联系非常紧的,前段时间百度把凯哥挖回来做多媒体组的负责人,那个组50多个人基本都是博士,说明百度非常focus多媒体的东西,因为现在用户的需求都上来的,简单的文字搜索满足不了多方面感知的需求。记得在百度实习的时候跟凯哥和圣君学长一起吃饭,深感自己的渺小,凯哥的一些话句句犀利,当时就留在国内还是出国留学的问题请教了凯哥,凯哥毕竟是在国外业界混的很好的人,不过还是那句话,在国外混的好的人真正体会一个来自其他国度的人在老美混的艰辛,其实国外的学校自费的master是不难申的,\textbf{学分绩高经历好的话完全可以冲下前几的学校},但是PhD就是难上加难了,考验全面素质的非常多。总的来说,在美国搞CS的可能不愁吃不愁穿不愁房子不愁车,但是如果想在公司得到提升是难上加难的事情;在国内混需要很强的人脉,这些都是可以通过时间和自己的努力建立起来,而且国内大公司对于人发展potential的挖掘还是比国外好的。凯哥提到了国外有两种人适合混:一种是无欲无求,享受其他的文化和生活的人,可能他不需要很多钱,自己经常到处逛逛就是种享受;第二种是国内就是超级大牛,这样你出国远远比别人高一截,外国公司不提升你完全没天理。这种人应该就是凯哥这种,哈哈。\par
 


下面说一下最重要的motivation。何谓motivation?我理解的是主动性,你有一些desire去做一些challenging tasks,比方说你想让一个教授做你的导师,我觉得最起码的是把他当前的研究稍微浏览一下,可能论文有些东西没有基础比较难读,可以挑一些简单的介绍性的书籍来读,这个过程是提高自己的过程,也是比较累的过程,因为这并不是我们普通考试给你范围让你背诵那么简单,你需要提出你自己的观点,更重要的根据你喜欢的领域或者认为比较promising的领域写一些proposal,非常推荐大家用\LaTeX{}写,这个东西排出来的版非常的美观,原来感觉比较难看的东西往上面一放感觉赏心悦目,这块东西也是OS实验课给大二的童鞋们写home assignments的时候练出来的,同时也感谢激动哥给了这么一个precious的机会。写proposal是比较有挑战性的,这时候建议大家看论文的时候不要放过任何的"all of a sudden",就是让大家突然出现idea,突然出现灵感的时刻,因为这些往往意味着一个可以去做的课题。而且如果有什么问题的话尽快与教授交流,教授都是非常happy与我们交流的,交流的时候最主要的是礼貌,另外把自己的问题阐述清楚。总体来说我在两个星期内完成了两篇research proposal,因为以前也没有写过,第一篇写的时间比较长,教授说问题比较general,没有真正去细说。这时候不能放弃,一定要再去纠正,当时在初面+笔试(笔试就是四道题目,就是离散+算法题,2个小时,有些题目比较有难度,当时觉得自己答得很happy,都做出来了)之后,还有第二轮的Skype面试,比较general是教授第二轮面试的时候提出的问题。对于教授的每个问题我们都要做有心人,必须要motivate自己去解决相应的问题,因为前段时间看过KDD 12的一篇oral的论文,并且把那篇论文的源码给读了,突然一些灵感的浮现让我觉得可以用一些第一篇research proposal提出的technique来解决。而且换另外一种方式来看这篇论文的问题又有一些新的灵感。这样在整整10个小时我完成了第二篇针对具体问题的research proposal,这篇一共写了11页,写完后感觉蛮累的,不过心里是非常开心的。我觉得就是这样自己step by step努力地提高自己,为自己争取一些优势,因为提前批的难度非常大,都走到这个时候了, 没有理由不为自己搏一把,搏的时候前面所做的积累,包括latex,包括machine learning一些理论的分析,包括前段时间看的那个源码,还有上一个星期的general的proposal,这些优势完全都在新的RP中进行了体现,到了这一步,这是真正让我自己为自己自豪的地方。\par
 
我们需要坚持自己么?其实旁边有非常多的案例,在学分绩和自己的爱好之间如何做一个trade-off。旁边不乏很多非常重视学分绩的,也不乏很多完全放弃学分绩的。我的建议是在学分绩有保证的前提下尽量做一些自己喜欢的事情,一般来说85是个坎,所以如果大家希望申请比较好的学校的PhD,还是要比85高的。其实港大下来我真实的感觉就是学分绩并不是那么重要,因为基本与教授们谈论的话题都与学分绩无关的,其实大家想想也知道,教授不会因为你哪科考的很好或者很差的就很看好或者看差你。我觉得HKU这方面做得很好,其实HKU是我儿时的梦想,一个综合性的大学去看人的眼光不会仅仅盯着你的某一方面不放。\par


很多事情都很有用,只是没到他该发光的地方。当时没日没夜跟着PGM课没用么?PGM是我率先与教授打开话匣子的地方;当时费了很多功夫讲OS实验课写材料没用么?这些完全锻炼了我,包括对一些事情的处理能力,\LaTeX{}工具的使用,以及交流时候如何更多争取。一步步走来,当你发现你是在不仅仅限于平时的学习,平时的考试的时候你才会了解到自己原来在一步步突破着自己,试着做一些比较hard的事情,让自己变得不仅仅局限于学分绩,更重要的是,让自己真正的强大。也许短期内看不过结果,也许短期内十分苦逼,我一直有信念,虽然有得必有舍,但是我相信我得到的东西一定会比我舍弃的东西让我更开心,而且我相信在我人生道路上真正让我proud的东西是我选择得的东西。借以最后一段希望大家平时尽量地不趋于安逸,遇到一些比较棘手的事情尽量挑战自己,在学分绩和自己的兴趣之间建议大家在不严重影响学分绩的情况下坚持着自己的兴趣。因为除非你是直接工作,要不然学分绩在申请中占得比例是蛮大的,虽然不是decisive,但是你需要用你其他的东西来弥补或者完全战胜你因为你的坚持而损失的学分绩。\par

最后,此文是我内心当前阶段真实的感受,想跟大家分享一下。我觉得在提前批拿到港大PhD的全奖有运气,但是更多的自己不懈的努力和争取。我觉得运气可以理解成probability,任何事情都带着运气的成分,但这并不是决定的因素。未来的道路还很长,我觉得我会用对得起自己的方式好好走向以后的路。大家完全可以带着批判的角度去看此文,还希望我的感受中有些能够对大家以后的申请带来帮助。有什么问题大家可以及时问我,我有时间的话会耐心给大家回复交流的。