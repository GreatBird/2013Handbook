\section{徐达博 CMU}
申请方向:MS in CS + MS in IS(Information Systems)\par
申请结果:
\begin{itemize}
\item Admission:MIS@U Az, IST@PSU, MISM(BIDA)@CMU, CS@UCD\\, UCSD, TAMU
\item Rejection:CS@U Wisconsin-Madison, UPenn二轮, USC, \\MIMS@Berkeley (被Big Dream秒了)
\item WL: 哥大
\end{itemize}
\par
一些条件:
\begin{itemize}
\item Revised GRE:Q170, V155, AW3.0
\item TOEFL:105(23)
\item GPA(xx/100):85+
\end{itemize}
\par
For futher information, especially about CMU-MISM, you can contact me via: dudnju@gmail.com

\subsection{把一直想说但是一直没有头绪的胡言乱语当做前言吧}
当我问自己我的大学是什么?我想是从热血到冷静到再热血的过程。柯景腾在小说里说,一场名为青春的潮水淹没了我们。大一大二,我一直在热血着出国,热血着GPA,热血着自己喜欢的妹子能喜欢我;大三一年,我冷静地想清楚自己到底想做的是什么,冷静的想出国对于我到底意味着什么;大四上,热血地在Davis周周跑去Prof. Ludaescher的Office Hour套近乎,热血着跟着Saumen做ProvEX的数据库开发,热血地帮Jim做Volunteer,最后拿到两封Reference Letter。\par
记得那本被我翻黑的老红宝书,记得大二夏天那惬意的北京新东方,记得大三一周之内大战铁血战士(G二战)和异形(期末考试)的苦逼,记得在梅花山了解到大四出去交换也没有想象中那么难,记得12Fall在Davis度过的美好时光。这一切,谢谢当初选择了出国,谢谢在这四年中我生活里的每一个人。\par
从大一立志于金融工程到大二的CS PhD梦,到大三发现应付了事的我是不会想熬到PhD的,再到后来通过信管的李学姐MISM-BIDA这个方向,再到在Davis的时候一遍遍改PS的过程中不断地给自己洗脑,这一路走来,我的目标偏离过,迷茫过,最后算是有了个最好的结果。我不敢保证十年,甚至三年后我会不会对今天的决定后悔,但是正如我自己说过的:“没有目标的人生,是一盘沙子,而全是目标的人生,好像自己又变成了被编译过的代码一样。我最喜欢的,就是告诉自己,大概就是往这个方向走吧,没有明确的目标,只是那条路上的景色看起来不错。”\par
这也是我在收到CMU的MISM通知后,放弃依然Pending中的USC, UCSD, UC Davis, Columbia, UPenn的原因。虽然挣扎了几天,虽然害怕以后找工作的机会没有CS出来的多,害怕不在加州没有找工作的地理优势,害怕赚的钱没有CS出来的人多,害怕以后只能在Tech方向上打辅助。但是这个项目是自己第一次看上去就怦然心动的项目。对做网页或是App都无爱的我,最喜欢的就是当数据告诉我,我的直觉是对的时候,那仿佛知晓了上帝的秘密的感觉。所以当再次去Heinz的主页看到那句“Students in the MISM-BIDA program acquire the skills to integrate cutting edge information and analytic technology practices with applied business methods.”的时候,我再一次心动了。皮卡丘,决定就是你了。于是交Deposit,准备寄材料给OIE。同时,和我深爱的加州说再见。\par
\subsection{申请中最出彩的地方是什么?}
这个是一个学弟问我的问题。申请中我最出彩的是什么?说实话我也不知道是哪一点让审我材料的教授就砰然心动了。但是对于我自己感觉来说,申请当中经历最丰富的,帮助最大的,当属在Davis的9-12月吧。\par
\subsubsection{关于课程}
9月一去,选上了最想选的Database Systems,虽然SQL部分已经学过,但是后面新的Datalog语法和每周一个Project的Deadline实在让学第二遍的我都觉得十分Challenging。坚持每周去任课的Prof. Ludaescher(鲁大师)的Office Hour问问题,给他留了个勤奋好学的好印象。当然鲁大师个人也是非常Nice的,Davis做数据库这块的好像只有这一位。另外,选了一门英语写作课,任课的Jim人超级好,我课堂表现也比较活跃,每次Jim要Timer给大家的上台Presentation计时的时候,我都毫不犹豫地举手Volunteer。所以每周Jim的Office Hour我也总能第一个去霸占30min-1hour和他一起修改我的PS。\par
最后三门课程拿到了3.9/4.0的GPA,虽然UCD的课很水,大家分数都很高(难怪UCD对于申请者的GPA有着近乎BT的要求),但是表面上看过去,还是很让人赏心悦目的成绩单。\par
\subsubsection{海外战T}
10月初海外第一次考T,悲剧的99。比国内一战高了一分,果然玩的太High了。然后10月份基本就在图书馆做作业和周末放假玩起来的节奏中度过了,10月份去了Berkeley,深深地爱上那里了。可惜我待Cal如初恋,Cal虐我千万遍T\^T。11月初考了第二次T,感觉无比的糟糕,尤其阅读一道单词题最后还把对的改成错的了(推荐无老师镇魂单词,我就考前看了从A开始的20个单词,第二天就考到这个了= =),幸好最后借小宝的寿星光环庇佑,拿到了105(23)的一般般分数。不过上一百了,就有敲门砖了。还有在海外考的好处就是和你一起的许多是韩国友人,印度有人和南美友人(至少我在的那个Fair Oaks考点是这样的!),所以你的发音算是灰墙好的那种,23分不错啦,对于我这种口语捉急的人来说。每次T我都没有认真准备,基本是考前准备模板的半裸考状态,所以最后105真的都是靠运气,还希望大家吸取我的前车之鉴,早点考GT,早点刷GT。\par
\subsubsection{文书工作}
文书包括了PS和CV两个,PS主要阐述Why 这个学科, why 俺们学校, why 俺们系这三个问题,当然不能直白地回答,要艺术性和创造性地结合自身经历,让看材料的人觉得你天生就是为他们这个项目生的。其中要非常感谢Davis的Jim和SASC帮我改PS语法和用词;要非常感谢已在MISM就读的信管李姐姐在忙碌的学习中帮我改了PS和CV,特别是李姐姐的点睛之笔对我在内容的改动上的启发,非常受用;谢谢楼和小卷的文书互助三人组,你们的回复让我知道非专业的人的阅读感受。PS和CV一定要给Native Speaker改,最好是面对面一对一的批改,如果不在国外的同学可以找一些英语培训机构看看,因为一般来说他们都会有此类服务。\par
\subsubsection{交换生活}
如果说以上是交换生活给我带来最直接的裨益的话,那么在UC Davis对我最大的帮助就是让我有了提前融入美国学习的机会。和Roger一帮人去Sky diving,和符学弟,死党黄毛晃荡了一圈加州。在Davis,每天上课,做作业,泡图书馆,缠着老师问问题,回家自己做饭,再用点时间考虑这个周末是坐Amtrak去Berkeley玩玩呢?还是去买国王队主场的门票看看NBA?亦或者是为UCD Aggie的体育活动加油助威拿免费的T-Shirt。连俺们交换项目的头头也说,This transcript cannot play a great role in your application. 现在自己回想起来,是在Davis每一天的生活让我越来越喜欢美国,激发起申请的动力。这里还要赞一下当地的华人教会,有信仰的人都十分善良。\par
\subsection{关于一些基本要素和硬件}
申请环节上有许多方面需要注意,我罗列一些基本的要素:
\begin{itemize}
\item GRE, TOEFL, GPA
\item Personal Statement, CV
\item 研究经历、实习经历
\item 选校+Deadline+投递材料
\end{itemize}
\par
我认为硕士的申请,最重要的还是你的GRE,TOEFL和GPA。这三者的重要性在一个档次上,同时基本上硬件相同的人,拿到相同档次学校Offer的概率也差不多。只要去多看看一亩三分地里看看那些报结果的贴,你就可以大概明白什么样的硬件能有什么样的档次。当然,关于硬件的说法并不绝对,你可以去陶瓷教授,你可以使出浑身解数去获得AD,比如这位GPA只有80的仁兄 \url{http://www.1point3acres.com/bbs/thread-56303-1-1.html}。我这里指的只是像我一样,没有什么很强背景,只想拿个硕士找工作的人。对于我们这种人来说,硬件是你的敲门砖。我这个总结就是给一亩三分地打广告的,CS|MIS申请,这个论坛提供的信息非常多,非常详细,质量非常高。所以很多细节我就不说了。接下来结合个人经历扯扯吧。
\subsection{关于俺这种非CS主流去申MIS的同学}
一亩三分地里面关于MIS方向的介绍应有尽有。但是我并不鼓励大家都不申请CS的方向转而申请MIS,因为对于中国人来说申请MIS和CS最后还是殊途同归要去做Tech,而往往CS在就业这方面占了巨大的优势,比如就有得到了UCSD CS和CMU MISM AD的同学在纠结到底要去那个。说实话,我相信SD CS的就业比MISM好,因为地理位置好,UCSD的名气也高,而美国的IT最大的市场就是湾区了吧。而Master项目的出路就是为了找工作。\par
那么我为什么还要申请MIS呢?因为个人非常喜欢CMU这个项目,觉得他就是教给我想学的东西。至于Berkeley的项目是因为两年,特别适合找工作。反观CS,深知自己会给大神们当炮灰,所以自己选的名气最高的也就是哥大,剩下UC系列完全是因为自己非常喜欢加州的生活Style。\par
这里稍微扯一下中介,我觉得中介可以找,但是不能什么都帮你做,我找中介也就是问问俺的选校名单科不科学,帮我寄了下WES成绩认证的材料,你的申请必须你自己做!想到一个自己都不熟的人帮自己编PS,做梦都会被吓醒!\par
 对于CS related转申MIS的同学,我建议你们在申请MIS的同时,千万不要忘记去申请CS的学校!和CS相比MIS的好处就是Broaden your horizon, 让你除了技术还有更多的相关知识,这和软工的知识有异曲同工之处,就是你要有了丰富的工作经验之后回去看,才能对这些知识有深刻的体会。那么CS的好处就是你继续相关专业的学习,有很多时间可以准备面试,一般的CS项目是两年(CMU很多奇葩时间就不算啦),有充分的就业时间准备。所以归结起来,技术是你的武器,无论什么方向,都只是一个平台而已,机会要自己争取。
\subsection{我在申请中的遗憾}
最大的遗憾莫过于错失了Berkeley的项目,不过后来看地里面的信息发现,这个项目的要求实在太高了,自己被秒杀是十分正常的。等Offer的时候,一定要放正心态,对自己有个全面的认识,不以物喜不以己悲,天天刷邮箱会越刷越焦虑的。不如干一点其他的事情,比如像我这样服务大众的,也可以做做实习,参加校园招聘积累经验Blah Blah之类的。\par

还有一个遗憾就是没有和其他同学共享选校的信息。首先,我非常希望软院出国的人越来越多。其次,通过建立考试互助小组的经验让我相信同学之间的不应该只有竞争,还应当有合作和共享,特别是合作共享要大于竞争。当然,人不为己天诛地灭。但是因为一个学校录取很多NJU SEer的可能性很低,所以基本上某学校如果我们院申请的人多,那么我们还是要在内部选出一个来,这样的情况下大神拿到AD的概率就是非常大的。但是,大众情人校又只有那么几所,所以呢,对于普通申请者(Rank 4之后,GT一般的),我建议如果这所学校实在想去,那么一定要申请!因为这是你的梦想,你要相信梦想的力量会让你的文书熠熠生辉,亮瞎Committee的狗眼。比如我的PS就是以CMU的项目开始写的,其他学校也是换换名字就投了,当CMU的PS写完,我自己通读一遍的时候我都快哭了,CMU你不录取我简直天理难容;比如我为CMU录了Video Essay(谢谢陈导的拍摄水平,非常高!),那自信,那神采飞扬,那改变世界的大手(不好意思,老罗广告植入了,为锤子OS打打广告啊未尝不好啊)。但是如果你仅仅是因为学校的名气之类没有很打动的你地方就申,那么你给大神当炮灰的几率是很大的。对于大神们,我相信你们肯定是申请的赢家,好学校都应该勇敢去申请,因为你是我们学院在申请者中最优秀的人,但是希望收到多个AD后,不想去的学校可以早点Withdraw掉,让WL上面的同学不要等太久,同时一定要推荐对方录取自己学院的同学,比如我的withdraw模板里的:
“I am confident that your department will be very successful in recruiting new students. Also, I hope that you could give favorable considerations to other Chinese applicants, especially those from the Institute of Software, Nanjing University, which is the best IT related Institute in China. ”

\newpage
\subsection{写在最后还是毫无头绪的话吧}

大三的一段时间极度迷恋小众音乐和装逼态度,后面居然因此歪打误撞捡了个天上掉下的陈妹妹。从此开始收心走上了纯洁的道路。虽然偶尔还是听听李志、Gala神马的,但是不再像当年那么emotional了。\par
我的世界里面充满了各种大神。夏总,在XDF教我写作,是原来南大外院前不见古人后不见来者的大牛,世界英语演讲比赛冠军,夏总鸡汤带给我的是对于世界的重新思考和内心的重新审视,从此给自己的人生要求就是做一个有趣的人。软院辩论队里面的各种大神。当年一道算法考题我纠结了好久施大神看一下题目,就点拨我如何去解。何姐更是各种神级的代表,属于传说级别。苏甦学长不说了。周博学长是去了USC笑看我等不入流的MIS,王馸学长在皇家科学院orz。还有我们这届的各种大神我都不敢写出来了..\par
 大神们的事迹可以告诉你,原来人生可以活到这种高度。但是你自己的生活却是你要一步一步走下去的。你可以为游戏里面的经验满槽花上一天一夜,你可以为金钱出卖自己的朋友,你可以为你爱的人不爱你伤心难过;但是,你也可以去学习你觉得散发着哲学智慧的东西,去和狗血欢乐的兄弟们打球扯淡、开车去膜拜Bunny Ranch这种绝逼了的地方,去等待那个你一直期望的灵魂契合的人。如果你现在都不敢热血地去做你渴望去做的事情,那么你敢期望那个四十岁,地中海,啤酒肚,每天朝九晚五,回家吃完饭就在电视前面等着睡觉的人去做么?\par
祝每个人的宠物小精灵都能到99级,打败小茂。\par
