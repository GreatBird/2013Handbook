\section{徐达博}
背景:
\begin{itemize}
\item Revised GRE:Q170, V155, AW3.0
\item TOEFL:105(23)
\item GPA:85+
\end{itemize}
\par
申请结果:
\begin{itemize}
\item Admission:U Az, PSU, MISM@CMU, UCD, UCSD, TAMU, USC
\item Rejection:U Wisconsin-Madison, UPenn, Berkeley, Columbia
\end{itemize}
\par

\subsection{我的背景}
申请中最出彩的地方是什么?这个是一个学弟问我的问题。说实话我也不知道是哪一点让审我材料的教授就砰然心动了。但是对于我自己感觉来说,申请当中经历最丰富的,帮助最大的,当属在UC Davis的9-12月吧。\par

在戴村我最后三门课程拿到了3.9/4.0的GPA,虽然UCD的课很水,大家分数都很高,但是表面上看过去,还是很让人赏心悦目的成绩单。此外感谢和我一起考试的韩国友人、印度友人和南美友人的口音,我拿到了105的托福分数。不过每次T我都没有认真准备,基本是考前准备模板的半裸考状态,所以最后105真的都是靠运气,还希望大家吸取我的前车之鉴,早点考GT,早点刷GT。

\subsection{文书准备的一些建议}
文书包括了PS和CV两个,PS主要阐述Why 这个学科, why 俺们学校, why 俺们系这三个问题,当然不能直白地回答,要艺术性和创造性地结合自身经历,让看材料的人觉得你天生就是为他们这个项目生的。一定要让Native Speaker帮你改语法,让同专业的学长学姐帮你把握PS结构。

\subsection{关于MIS}
实际上,我并不鼓励大家都不申请CS的方向转而申请MIS,因为对于中国人来说申请MIS和CS最后还是殊途同归要去做Tech,而往往CS在就业这方面占了巨大的优势。W大总结的很好,CS人多机会多,MIS人少机会也就少一些。\par

那么我为什么还要申请MIS呢?因为个人非常喜欢CMU这个项目,觉得他就是教给我想学的东西,此外可以旁听些感兴趣的CS课程。反观CS,深知自己会给大神们当炮灰,所以就没有再念CS了。\par

对于CS related转申MIS的同学,我建议你们在申请MIS的同时,千万不要忘记去申请CS的学校!和CS相比MIS的好处就是Broaden your horizon, 让你除了技术还有更多的相关知识,这和软工的知识有异曲同工之处,就是你要有了丰富的工作经验之后回去看,才能对这些知识有深刻的体会。那么CS的好处就是你继续相关专业的学习,有很多时间可以准备面试,一般的CS项目是两年,有充分的就业时间准备。所以归结起来,技术是你的武器,无论什么方向,都只是一个平台而已,\textbf{机会要自己争取}。

\subsection{一些补充}
我非常希望软院出国的人越来越多。其次,通过建立考试互助小组的经验让我相信同学之间的不应该只有竞争,还应当有合作和共享。但是因为一个学校录取很多NJU SEer的可能性很低,所以基本上某学校如果我们院申请的人多,那么我们还是要在内部选出一个来,这样的情况下大神拿到AD的概率就是非常大的。所以在申请过程中的互相通气蛮重要的。

我的邮箱:dudnju+applyforus@gmail.com
