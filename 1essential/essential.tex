\chapter{申请的硬性条件}
\newpage
\section{核心的核心--GPA}
本节主要讨论下我们申请最重要的硬件,GPA。
\subsection{关于GPA}

GPA是整个申请过程中最重要的一个数据。其重要性超过了GRE,TOEFL。所以对于GPA来说,永远没有够不够一说,越高越好。尤其是对于一些控GPA的学校(比如Yale, Upenn等),这些学校把GPA看得很重,当然在衡量申请人的GPA的时候,绝大部分学校回参考你的本科学校的情况,比如北大的85分,就要比二本学校的90要更有竞争力,所以本科出生也是一个很重要的因素。当然,南大的学生,在本科出生上就有了很多优势,学校在国际上的声誉一直不错,所以在一开始,你们就已经领先了其他学校(不包括清北上复科)的申请人。不过这一点优势并不能弥补你过差的GPA,提高GPA还是非常必要的。

\subsection{如何提高GPA}
\begin{enumerate}
\item 当然是认真学习。这是我认为最好的一种。GPA其实是你自身学习情况的反应,学的好不好一定程度可以从GPA上反应出来(不过也有学的很好,GPA不高的情况)。其实最终对你而言GPA什么也不是,只有学到的知识才是属于你的东西(有些学校也意识到这一点了,所以会认真参考申请人的各项数据和申请材料来确定申请人的学习情况和将来的潜力,比如Brown,UCSD等)。

\item 选水课。我们院的水课还是有的,也可以从这一方面提高GPA。水课就是那些给分很高,又不用花太多精力学习的课,当然这种课上收获的东西也比较少。所以选水课并不是一种特别推崇的方式。因为水课选多了,你很有可能成为水人,在技术上和知识上都不精,可能GPA比较好看,但是实际外强中干,这种情况在出国读研的过程中,会显得非常吃力(我还没去,这只是我的猜测)。当然变态课还是要慎选,变态课是指那种认真学习了,各种项目也做了,结果莫名其妙会的很低的分数的课,具体是哪些课,最好私底下问问学长学姐,这就不公开讨论了。

\item 注销、重修。这也是一种提升GPA的方式,但是十分不推崇,费事费钱的。而且注销和重修尽量要在大四之前完成,那就是说要在大二和大三时期完成重修工作,但是对于软院同学来说,大二大三是你们最忙的时候,去重修的话,很有可能时间安排出现问题,导致重修的成绩还没有以前的成绩好,这一点非常痛苦,如果你选择重修的话,不管最后重修成绩是多少,都是按照重修成绩记录,以前的成绩会被覆盖,所以重修有风险。
\end{enumerate}

GPA想说的大概就是这几个方面了。再说一些题外话,GPA只是个数字,盲目推崇GPA其实在实际工作和生活中毫无意义,真正重要的是GPA下面的东西,就是你学习的情况,应试对于大学生不是一种很好的学习方法和学习态度,这一点很重要,学会主动学习,学习本质的东西,才是最根本最有意义的。当然,大学生活中学习不是唯一的内容。大学是个很不错的环境,能够获得很多东西。参加学生活动,认识新的朋友,是很不错的选择;出国交换也是一个很有意义的事情,不管去哪里,去北美,去欧洲,去香港台湾,甚至去非洲支教都会是一个不错的选择,开阔眼界,了解不同的文化,这一些虽然不会提升你的GPA,但是会对你的三观有很大的影响。总结来说,GPA很重要,越高越好,但是为了GPA放弃大学生活,那你就输了。

\section{必要的英文成绩--TOEFL/GRE/IELTS}
本节主要涉及托福/雅思成绩和GRE成绩在申请中的作用。笔者将GPA、TOEFL等成绩视为硬材料;将PS、推荐信等视为软材料。
\subsection{托福没上100就会死?}
很多人都说,托福100、旧GRE1350是道坎。然而实际情况是It all depends。诸位申请者需要明确的一点是:无论申请MS还是PhD,英文能力测试成绩没有GPA来的重要,但是你的英文能力需要有\textbf{一定的水平}。\par
不同的学校对于托福的要求是不一样的。笔者拿三个项目作为例子:
\begin{center}
\begin{tabular}{|l|l|l|}
\hline
学校 & 项目 & TOEFL要求\\ \hline
CMU & MS in eBiz & 86 ibt\\
哥大 & MSCS & 100 ibt\\
S大 & MS CS & 113 ibt\\ \hline
\end{tabular}
\end{center}\par

从上面的表格中不难看出,不同项目对于托福的要求是不同的。对于CMU的eBiz项目而言,申请者只要86分以上,申请材料应该就不会因为托福的分数不够而遭到拒绝。随着使用雅思申请的申请者越来越多,美国高校接受雅思成绩的也多了起来。对于美国申请而言,雅思和托福的情况类似。笔者建议申请者在选校的时候要查看清楚目标项目对于托福、GRE、雅思的要求。有的项目会在项目描述中写出,有的则在FAQ中,一般搜索一下,在学院网页上多浏览下就能找的到。如果本院的申请者们能够在出国群中共享这些项目的硬性要求,相信能够节省每个人很多的时间。\par
这里虽然只是讲了讲托福的问题,但是实际上雅思和GRE都是类似的:不同的项目有不同的分数最低要求。只要过了最低要求,那么你的申请文书等其他软材料理论上是会被仔细查看的,但是不排除排在你前面的人太多而被刷掉的可能性。此外,即使没有到最低要求,你也可以询问下小蜜差个1、2分可不可以申请。\par

在这里推荐W大的\href{http://www.1point3acres.com/\%E7\%BE\%8E\%E5\%9B\%BD\%E5\%BD\%95\%E5\%8F\%96\%E5\%A7\%94\%E5\%91\%98\%E4\%BC\%9A\%E5\%88\%B6\%E6\%8B\%9B\%E7\%94\%9F\%E6\%98\%AF\%E6\%80\%8E\%E4\%B9\%88\%E5\%9B\%9E\%E4\%BA\%8B/}{《美国录取委员会制招生是怎么回事》}这篇文章。相信读完这篇你就会对申请的流程有个大致的了解,也就知道软材料和硬材料在申请中的作用。

\subsection{什么时候寄成绩合适?}
讲完了坎的问题,我们来聊聊什么时候寄成绩单合适。当然,早寄早好。因为考试机构如ETS之流从接受你的申请到寄出信件一般要一个月的时间。如果目标学校和ETS已经有了网传成绩的协议,那么速度会快一些,但是这些信息作为申请者是不知道的,所以你能做的就是尽早寄成绩单,然后刷刷网申系统查看状态。如果在ETS查询到成绩寄出后一周,网申系统还未更新你的成绩信息,那么你需要赶快向小蜜确认下是否收到成绩单了。记住每次和小蜜联系的时候附上自己在网申系统的ID,这样会方便对方查询。\par
如果你的成绩在对方院校开始审理材料之后还未送到,一般Admission Office都会notice你的。你可以再请求ETS补寄一份同时先为对方提供在线的成绩单。这样做一般就不会有问题了。
\section{写出你自己--PS/CV}
本节主要讲述出国材料中文书写作的基本要求和注意事项。\par

文书(Essays)主要包括个人陈述(PS/SoP)、履历(CV),另外还包括写作样本等可选文字材料。文书在申请中的作用在于体现申请者的写作能力和经历的独特性,以便让committee对申请者形成一个全面的评估。申请过程中,除了GPA、GT成绩等硬实力,推荐信、文书这些软实力有时会成为左右申请成败的X因素,所以切不可用散漫的态度对待文书写作,切忌模板化、套路化。
\subsection{文书写作的一些原则}

1. 真实性。千万不能列举不实信息,integrity是老外非常重视的基本品质,不要试图挑战底线。分清事实和议论,你可以认为某个project上自己取得了很大进步,但不要编造类似“获得了某某奖项”这种不实信息。\par
2. 个性化。突出自己的优势所在,充分展示自己的能力,结合自己的经历阐述对专业学习的认识和职业目标的规划。\par
3. 符合要求。每个学校对文书的内容、格式、附加信息都或多或少有一定的要求,不满足要求的文书不仅不利于committee的正常评审流程,而且会显得“诚意不足”。

\subsection{CV}

CV,即履历,是一份对你整个人生经历的总结,涵盖了教育、工作经历、职业活动、发表/出版物及其他重要人生经历的综合性文档。CV同简历(Resume)不同(具体请参考\href{http://studentaffairs.psu.edu/career/pdf/CG/CG_Vita.pdf}{这个文档}),长度是没有限制的,所以只要学校没有特殊要求不要怕写长。\par

CV的目的在于在短时间内掌握申请人的整体情况,在申请者很多时可以快速进行评估,或以此为出发点来阅读PS;对PhD申请者,可以从发表论文和参加的学术活动看出其科研水平。\par

一般针对出国申请的CV包括以下信息:基本信息、教育经历、科研情况、项目经历、工作经历、获奖情况和其他信息。逐条来看:
\begin{itemize}
\item 基本信息:申请者的名字、联系方式等,一般在最开始部分。
\item 教育经历:列举自己的高等教育经历(包括交换项目),包括学校及位置、学习时间、所学专业、取得学位、GPA等。近期的经历在前。
\item 科研情况:列举所在的科研小组、实验室,参与的科研项目及承担的职责,发表的论文等。论文采用标准引用格式。
\item 项目经历:CS MS申请者可以列举若干自己参与的软件项目佐证自己的软件开发能力,信息包括项目名称、时间、简要介绍、承担的职责等。同样要按照逆时间序排列。
\item 工作经历:列举自己从事的正式工作或实习工作、助教工作,信息包括雇佣者、时间、职位、职责等。
* 获奖情况:列举自己所获的重要奖励、称号等,信息包括奖项名称、颁奖方名称、获奖时间等。
\item 其他信息:可以包括课外活动(比如志愿者、社区义工活动)、leadership(比如学生会工作)、软件技能、标准考试成绩等。按个人需要选取。
\end{itemize}

CV写作的注意事项包括:
\begin{itemize}
\item 使用模板。建议修改专业的履历模板而不是自己创造,除非你对自己的水平很有信心。
\item 排版清晰。Keep the reader in mind! 字体选取要恰当不能让读者感觉累,排版不能过于松散或密集。建议使用\href{www.latextemplates.com}{ \LaTeX{}排版}。
\item 突出重点。重要的事实采用粗体或斜体标出,论文作者中自己的名字用下划线或加粗标注。
\end{itemize}


\subsection{PS/SoP}

个人称述一般结合申请人自身的经历说明自己为什么选择特定专业、导师和学校,以及对研究生生活的规划。按风格可以分为PS和SoP两种。\par

PS,Personal Statement,有的学校也称Personal History Statement,一般是话题式的文章,对行文方式限制不多。一般申请者可发挥自己的想象力来确定文章脉络、整体结构、行文方式。一般不超过两页。\par

SoP,Statement of Purpose,相对于PS就正式得多,一般明确申请者阐述自己的研究兴趣和经历、选择某校的理由和对自己短期和长期的规划。发挥的空间相对于PS小很多,一般不超过一页。\par

当然,这里的PS和SoP之分是按风格来分,不同学校对个人陈述的称呼不同,有的挂SoP之名却行PS之实,所以希望申请者看清楚具体学校的要求,不要被名称所蒙蔽。\par

个人陈述的基本要点在于claim \& support。比如你不能平白无故说自己非常喜欢CS,这里面肯定有个认识的过程,比如你很早接触编程,或者被MS、Google这些公司改变世界的行为所折服,等等。其实说白了就是要注意逻辑,你可以写得非常有创意,但不能让人觉得内容incoherent。没有证据的观点就是废话。当然,support不一定直接从个人陈述本身来,你可以在CV中列举了参加的学科竞赛,然后在个人陈述里说参加了很多学科竞赛;如果你有nice的导师,有推荐信的佐证,那么你可以说自己科研工作开展很顺利。\par

SoP的写作也参考模板,略去不谈。PS的写作建议自己构思和组织材料,不要急着去参考往年的文章,只有你自己才最了解自己,先写再改,然后从别人的文章中吸取构思方面的经验。随着你构思的不同或角度的变换,PS可能会有多次大的改动,这是个好现象,说明你对自己的理解更加深入了,你越来越找到了阐述自己competitiveness的方式。同时,你可以准备几份个人陈述,以配合不同学校的偏好。比如重科研的学校你就给突出科研能力的陈述,重实践的学校你就说自己的项目经历多么丰富和成功,在加州、纽约的学校你谈谈自己的创业理想和实践\ldots\ldots让committee觉得你确实有不得不去这个学校的理由。同CV相似,文章中的重要内容可以使用粗体或斜体标注。\par

最后,相对于CV,个人陈述需要持续不断地修改。可以是自己改、互改或请专业的文书修改机构改。自己修改在文书写作早期有效,但成型之后的效果就变差,因为你对文章太熟悉了,会产生定式思维,所以务必请同学或老外帮你修改。至于要不要动用收费服务,就因人而异了。
\subsection{其他文书}

一般学校都要求申请者提交自己的CV和个人陈述(PS、SoP其一或都要),还有一些学校会额外要求其他一些文书。\par

写作样品(Writing Sample),一般是你写的课程论文或学术论文的片段,主要考察申请者的写作表达能力。\par

多元性声明(Diversity Statement),申请者需要用简短的文字表明自己能增进学校的diversity。一般从人种、文化、独特经历等方面展开。很少学校要求单独的文书,一般都写在网申的一个文本框中。

\section{别人眼中的你--RL}
本节主要涵盖的内容是推荐信Recommendation Letter(RL)的获取和自己写推荐信的误区。
\subsection{什么是推荐信}
推荐信是学校评估申请者使用到的材料,一般由答应为申请者推荐的人书写。申请者在网申系统中需要\textbf{waive自己查看推荐信的权利}。在推荐信当中,一般推荐人会陈述自己和被推荐人的关系,他为什么推荐被推荐人。内容完全由推荐人把握。因为网申早已普及的关系,现在推荐信一般是由推荐人通过注册自己的推荐人账号上传到网申系统中。\par

对于申请美国的硕士和博士项目,推荐信基本上是必不可少的。不同项目对于推荐信的要求是不同的。大部分是需要你提交三封推荐信,当然也有要求两封的学校。有的学校要求你提供规定数量的推荐信(比如只能提交三封),有的则是要求你规定数量或以上(比如三封及以上都是可以接受的)。所以对于具体的要求,需要看申请项目的规定。\par

\subsection{寻找推荐人}
在推荐信部分,申请者首先需要寻找到适合自己的推荐人。笔者认为,根据不同的项目,需要有不同的推荐人。所以在你的推荐人名单上至少需要4名人。针对学术性质的Master(MS)/PhD.项目,除了任课老师之外,可能需要一封涉及自己研究工作的推荐信,那么你的研究导师无疑是最好的候选人;针对面向就业的Master项目,一封实习期间主管或者Mentor的推荐信相信能为你加分不少。\par

在一个面向就业的Master项目中,Admission Office建议申请者按照如下的情况为自己准备推荐信:

\begin{center}
\begin{tabular}{|l|l|l|l|}
\hline
Work Experience	 & \multicolumn{3}{|c|}{Suggested Recommender Type} \\
\hline
0 Years	& Academic	& Academic	& Academic\\ 
0-1 Years & Academic & Professional/Academic & Professional\\ 
1-3 Years & Academic & Professional & Professional\\ 
> 3 Years & Professional & Professional & Professional\\
\hline
\end{tabular}
\end{center}

那么选择什么样的推荐人呢?很多人觉得大牛的推荐信能够很有分量。但是笔者认为,It all depends。对于面向研究的MS/PhD项目,学术界大牛的推荐信会很有用;对于面向就业的MS项目,学术界大牛的推荐信就要看大牛写的内容了。需要注意的是,这里的大牛指的是对方学校所认可的大牛,而不是你认为的大牛。\par
如果你的推荐人毕业于这所院校甚至你要申请的院系或者这位推荐人曾经在这个学校工作过,那么相信他即使不是大牛,其推荐信也会很有分量。笔者认为自己一个AD的确很大程度上归功于推荐人曾经是这所学校的研究人员。但是,一般我们都需要申请10-20所的学校,所以这种情况不太适用于我们的申请。\par
总而言之,大牛的推荐信是有用的。但是,不是每个人在前三年都能有幸在大牛手下干活并且和大牛有一定认识。下面我们分情况来讨论没有大牛和校友情况下的推荐人选择。\par
先来说说有志于学术项目的同学们。一般在前三年的时间内,你已经进入实验室并且跟随导师做了一些研究。那么你导师应该成为你的推荐人。此外,你获得高分的课程的老师也很适合做你的推荐人,即使这位推荐人可能只是Lecturer。此外,还有一份推荐信大家可以发挥自己的能力去取得了。\par
下面我们来说说就业项目的情况。在Academic方面,一门高分且和项目相关课程的老师可以当做自己的推荐人。在Professional方面,有实习的Mentor的推荐信足够了。此外还需要一封推荐信可以来自教授身份的任课老师。因为国外大学对于教授还是很尊重的。\par
总而言之,针对不同类型的项目,在推荐人上是需要有小小的区别的。因为不同推荐人在推荐信中看到的你是不一样的。

\subsection{如何获得一封好的推荐信}
这个单元我们讨论的是推荐人帮你写而不是你获得同意后自己写的情况。听说清华的老师都是自己亲自写,我认为这是很好的风气。这是推荐人对自己信誉的负责。\par
如何获得推荐信,更多考验的不是你的智商,而是情商。在大三结束的时候,申请者在心中就应该有一个初步推荐人的名单了。推荐人和你在一起共事的时间越长,关系越好,就越容易拿到他的推荐信,而且认识时间越长,其推荐信也更有作用。为什么认识时间越长的人推荐信越有作用?聪明的你一定知道。\par
如果需要拿到老师的推荐信,那么上课就要经常“帮助”老师解决问题没人回答的窘境。课间和老师聊聊天、课后参加老师的一些研究、当助教都是很好拉近老师关系的方法。对于去外国交换的同学,Office Hour一定要去,混脸熟绝对没错。\par

如果推荐人答应了帮你写推荐信。那么恭喜你,你做好了第一步。下面我们说说如何获得好的推荐信。一封好的推荐信是你期望推荐人展示出来的你的品质。这需要你自己委婉地提出你的期望。一个Tip就是在对方答应给你写推荐信后,你可以总结下你和推荐人一起做过的事情、你期望推荐人陈述的方面,并将其总结成一个statement并附上自己的CV一起email给推荐人。比如下面的例子。\\
\\
\emph{
"Dear Professor XX,\\
\\
Thank you for becoming my reference. I attached a short statement of my experience in the course xx and the xx project. I wish this could help you when you write the letter of recommendation.\\
\\
Best regards,\\
XX"
}\par
在\href{http://www.wikihow.com/Ask-Your-Professor-for-a-Letter-of-Recommendation-Via-Email}{WikiHow}中有关于如何获取教授推荐信的步骤,在WikiHow的这个问题中也涉及了西方文化中的书信礼貌等等问题,相信对于陶瓷的同学也会有帮助。


\subsection{自己写推荐信}
相信对于大多数人来说,大家还是需要自己写推荐信的。那么最重要的就是记住要\textbf{站在推荐人的角度上}写推荐信。全都是吹捧会让人觉得这封推荐信是fake的。此外,不要有语法错误。不同的推荐信在格式、字体等等方面也要不一样。\par
笔者在自己写推荐信的时候,采用了和别人互相批改的方法以防止推荐信看上去过度的赞美。在推荐信中,赞扬的形容词可以有,但是不能太多。中国学生自己写推荐信往往会坠入he is a nice guy的圈套。实际上,如果没有对于事情的描述,对于申请人经历的描述,这句话一点用处都没有。笔者结合自身经历总结了自己写推荐信的几个误区:
\begin{enumerate}
\item 使用太多的评价语句
\item 内容空洞没有逻辑
\item 语法错误、格式错误、Chinglish写法
\item 经历描写太过于细节
\end{enumerate}\par
总结完误区后,笔者推荐\href{https://www.e-education.psu.edu/writingrecommendationlettersonline/node/112}{这篇文章}中所提到的关于写推荐信的描述事项。\par
自己写推荐信另外要注意的就是和其他同学共享推荐者账号。在本节开头我们说到了现在网申基本上都是由推荐人在网申系统中注册账号提交推荐信的。我们的实际情况是一个推荐人往往答应了我们学院不止一个学生的推荐信请求,所以建议大家\textbf{建立出国群}来共享一个“推荐人”的邮箱、网申系统密码以防止明明是相同推荐人却在一个系统中拥有多个邮箱信息的情况。\par
此外,一些项目在录取后会对录取的学生进行Verification,而验证是否有这样一个推荐人往往是Verification中十分重要的部分。比如笔者就很“幸运地”被要求提供所有推荐人的电话号码方便验证机构调查笔者材料的真实度。所以有这样情况的同学一定要在验证之前知会推荐人并概述自己所写推荐信中的一些内容。