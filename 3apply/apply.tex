\chapter{开始你的申请}
\newpage
\section{选校}

选校简单说就几个步骤。
\begin{enumerate}
\item 在usnews网站上找到computer science的专业排名列表,按照排名画个范围出来,比如1-70。

\item 然后开始初筛。把你确定的范围内的学校的网站打开,看看每个学校招生页,了解每个学校招生的标准和要求,筛除那些不能满足标准的学校(有些标准是吓唬人的,如果你对某个学校非常感兴趣,但是学校给的标准很变态,你可以发邮件问问这个学校的小米这些标准是不是硬指标)

\item 通过第二步后剩下的学校可以进行第二轮筛选。第二轮主要通过主观因素筛选。即你对这个学校的印象怎么样,所处的地理位置如何,气候如何,安不安全等等(对特定学校的评论你可以在\href{http://www.1point3acres.com/bbs}{一亩三分地}上找到)

\item 这样可能剩下还有20所学校左右,再看看学费生活费什么的,极有可能你会再放弃其中的5所学校,当然(个人建议仅供参考)除非是非常喜欢和非常牛逼的大牛校,尽量选择东北部和西海岸的学校。

\item 最终剩下的学校在10~15所学校为最优。其中最优分布是冲刺2~3所,主申6~8所,保底2~3所
\end{enumerate}

最后,其实我想说,只要学校的career fair给力就够了,其他的都是浮云,没必要计较很多。关于加拿大学校,和美国申请的不同之处比较小。不同的地方有:部分加拿大的学校可能不要求GRE成绩,大部分的学校需要面试,学术性的硕士项目相对美国会多一些。此外就没什么差别了。

\section{中介}
只有体验过中介服务的人,才能告诉你中介到底在干什么。一家之言,姑且为学弟学妹当做参考。
\subsection{需求}
找中介的原因不外乎省事儿和申请到更好的学校这两个原因。当时,我的情况是GT分数不够,并且GPA也算不上很好,所以我想要找一个靠谱的中介,利用我考试刷分的时间就给我好好安排;同时,在申请开始后提供一个聪明的英文的老师帮我提升申请材料的质量。为了达成上述目标,我总共考察了十二家中介。不得不说考察过程还是有满意的地方的:
\begin{itemize}
\item 大多数中介服务热情到位,我问了很多问题,都没有一丁点儿不耐烦;
\item 咨询老师对我的情况分析细致,还据此提供了一些看起来靠谱的历史数据;
\item 从咨询的过程中,我从他们口中了解不少关于美国学校和不同地区的有用消息;
\item 咨询老师对可以提供的服务描述得很具体也很美妙。
\end{itemize}

但是不满意的地方也有:
\begin{itemize}
\item 中介不愿意将过去学生的材料展示给客户看,虽然这可能是出于隐私考虑,但是将关键信息删去以后展示给我看我认为也无伤大雅;
\item 根据我目前的GT成绩和GPA,咨询老师给我参考的能申请到的学校排名让我感到不满意;
\item 中介里没有CS方面的专门人才,大多对CS属于完全不了解,有了解一些的,但是也只懂些许皮毛;只有两个中介里有理工科老师;
\item 中介费普遍较高。
\end{itemize}

不论如何,最终因为种种原因,如此这般我就签下了中介并付了全款。当然,在此之前我又刷几遍了GT,并且放弃了继续刷的努力\ldots\ldots

\subsection{设计}
设计这一部分包括中介对我的前期规划和选校。前期规划就是填写中介给的一系列表格,包括自己的信息,还有申请材料的信息。这个时候规规矩矩地填写表格/问卷就好,虽然不是很顺利,但是还是觉得自己在一步一步地做事。选校则比较复杂了。通常是中介根据你的情况给你提供一个参考范围,你再从其中选择。虽然是让你随便选,但是中介会有如下不成文的要求和建议:
\begin{itemize}
\item 选校要有层次性,每个排名区间的学校都要选一些,女神校可以有,保底校也必须有,名次中庸的多选一些。
\item 除了排名之外,过去的录取率也是很大依据,去年全灭的不能选,去年录得少的少选,去年录取情况尚可地才能随意选。
\item 虽然中介会提供的是“参考”学校,但是如果你的选校列表里有稍微多一两所范围之外的牛校,也是会被他们一遍又一遍劝阻的。
\item 在劝阻地过程中,中介会推荐你选择他们认为很容易申请的水校。
\end{itemize}\par
总之,中介会努力让你选择中标率大的学校,而我就在他们的软磨硬泡中一遍又一遍地刷太傻、一亩三分地和各个学校的官网,来找出自己究竟想去什么学校。经过了一系列温和和强硬的交流之后,我定下来了双方都还算满意的选校列表。此时,我明白了,找中介省不了事儿,因为我想去更好的学校和中介想要我选更保险的学校这两点是完全矛盾的,所以我\textbf{需要把握申请的每一个过程},保证中介不借机耍滑。

\subsection{实现}
实现的过程包括寄成绩单、写材料、填申请。在这里到了一个三叉口:我可以选择全权由中介完成,或者中介只负责写材料和检查申请表填写是否正确。\par
作为一个不准备省事儿了的完美主义者,我自然选择后者。当然事实证明事情并不是那么绝对的,我还是在中介那里填了申请,寄了成绩单的。寄成绩单这是一项体力和智力的双重劳动,飞跃手册里别的学长会详细解释的。不过选快递、选邮寄时间和刷Tracking确实是一件略痛苦的事,在此略心有余悸。\par
在文书准备方面,中介能力缺乏,在写材料方面不仅没给我提供任何帮助,还添了不少乱。这里就不赘述了。填申请应该算最顺利的过程了。作为一个deadline驱动症严重患者,中介还是起到一定的敦促作用的;虽然与此同时至少有一半学校是卡在deadline之前最后一个工作日提交的。大部分申请是我自己填写的,然后中介检查以后提交。提醒一下,在这种时候,你的信用卡信息会提供给中介,不给还不行。\par

到此为止战斗结束,长叹一口气。不过读者童鞋,你也看出来了吧,中介一不懂你,二不懂你的专业,三不愿意动脑子,四不愿意动手。你觉得,他凭什么能帮你申请上比你申请的更好的学校?

\subsection{测试\&运维}

如果非得把一些内容放到测试这一部分的话,那就是我的申请结果。申请结果尚可,至少高高低低啥都有点。中介此时就是坐等结果了,把我的结果都记录在案就完事儿了。\par
接受了学校的录取之后,有各种各样的事情要做,包括找房子、买机票、体检、签证、收拾行李、开成绩单等等。中介此时只问我关于I-20和签证的事情,其他都没管。这我也能理解啦,因为毕竟他们只保证把我顺利送上飞机,没保证让我顺利在美国settle down。此时推荐参考你将去的学校的新生群和一亩三分地。

\subsection{客户反馈}

作为一个客户,我能给出的反馈。也就是下面几句话:
\begin{itemize}
\item 需求:一般满意;选中介其实挺纠结。
\item 设计:较不满意;选学校一定要仔细了解亲力亲为,不要偏听偏信。
\item 开发:不满意;如果这个世界上有人了解你,并且愿意为你的理想付出一切,那一定是你自己。BTW,好好学学,咱文笔不见得比那些英语专业毕业的童鞋差。
\item 测试:较满意;这是我自己努力地结果。我尽力,所以我知足。
\end{itemize}

这一届找中介的不止我一个,抱怨的也不止我一个。我只是作为吐槽代言人而已。就像另一个童鞋说的那样,中介就是光收钱不做事,还不如直接找新东方老师写文书。在我看来,这货性价比太低,咱家贫人贱的,真真是白瞎了。适用人群仅限:完全不想管出国的一干事宜,\textbf{只要能出国就行,对于上什么学校也不计较},并且觉得自己在文章里无话可说的白富美高富帅。\par

误区解读:
\begin{enumerate}
\item 和中介签的合同并未规定提供什么质量的服务,写出什么质量的文书,这种充满形容词修饰的东西,不要相信。
\item 我在中介里有多个老师为我服务,也有一个老师对我的情况很了解,但是没有人会为我负责。因为责任界定很模糊。
\item 在我曾经设想中,在我忙事情A的时候,中介可以帮我准备事情B。但事实上,这是不可能的。他们是“严格地”一步一步来的,你不做A,中介就不会准备B,如果拖到了最后,估计他会算你自己的责任。
\item 在咨询的时候我听到了不少“小道消息”,我曾经觉得这就是他们的卖点、他们的资源。后来明白:a)这些消息都是过期的,并且来源都是网络论坛什么的——也就是说,你自己也能搜到。b)这些消息有用的太少,因为需求不同,准确性难说,所以它们通常影响不到你的申请决定。
\item 在中介里,不同阶段都有不同老师搭理你,之前的老师几乎完全不理你。我怎么看怎么都觉得他们是术业有专攻。因为忽悠人的只会忽悠人,写文书的不能乱说话。
\item 曾经加过一个中介老师的企鹅号,名字是XX留学Y老师。他声称自己是普度CS博士,回来觉得做中介挺好,能够帮助别的同学,将会一直做中介。半年后我发现那个企鹅号已经换成了Z老师。祝在他手上“颠簸着”的同学申请顺利一路平安。
\end{enumerate}


\section{网申管理}
申请管理的确是个技术活。在整个申请过程中,有各种各样的数据需要记录和查看,为了高效的完成申请工作,对申请数据的管理显得尤为重要。

\subsection{申请前期管理}
   申请前期的数据主要是选校分析数据,包括学校的基本信息,项目的基本要求等。\par
   这个时期的数据来源主要是学校的官方网站,一般在网站的prospective student栏中。此时需要搜集的数据大致为,学校的综合排名,专业排名,GRE,TOEFL,GPA的基本要求(有些学校还给出了往届录取的平均分,比基本要求更具参考性),申请文书(包括PS,SOP,RL等)的特殊要求等。同时,也建议了解一下学校的学费和生活费情况,以及每届招生规模等。这些数据有助于为你选校提供切实有效的帮助。申请的同学可以组成一个小组,共同搜集和整理申请前期的数据,在这里建议小组使用Google Docs。因为前期的数据主要为学校的基本信息和项目的基本要求,这些数据比较零散和固定,分工合作,可以减少个人的工作量,同时提高数据的准确性。


\subsection{申请中期管理}
   申请中期数据管理就是个人工作了。主要用于记录申请进展,包括成绩单邮寄情况,GRE,TOEFL成绩寄送情况,推荐信提交情况,申请状态查询网址,申请结果,备注等信息。将这些数据按照学校项目为条目记录在excel表格中,可以系统的追中各个学校的申请进展。避免各个项目进展混乱等情况。

\subsection{申请后期管理}
   申请后期管理主要是针对已经确定学校的同学,记录自己后期飞跃的进展状况。这个时期的工作就比较简单的,一般对方学校会给即将入学的新生一个checklist,同学们直接根据学校给的checklist就能很好的管理自己的飞跃进展了。

\subsection{友情提醒}
对于所有的纸质材料,最好都扫描一份存在自己电脑里面,很多时候学校允许使用扫描版文件就会省掉大额的快递费用,另外像护照,录取通知书,等的扫描件在申请后期也会有很大的作用。存储一份证件照的电子版也会有很大的作用和便利。


\section{学会沟通}
在申请过程中,若遇到任何问题(比如申请材料要求不明、付款出现问题等),除了参考其他人的经验,免不了需要同小蜜进行沟通。沟通的方式一般是email和电话,以email为主,电话为辅。除了问小蜜问题,和教授陶瓷也类似,一般交流采用email,面试则是电话(以skype为主)。
\subsection{沟通前的准备}

首先,\textbf{请务必阅读所申请学校的招生FAQ},确保你的问题不在FAQ中,以免浪费自己和他人的时间。\par

其次,最好上论坛看看,看看你的疑问是不是已经有人帮你问过了。一般你遇到的问题,别人也会遇到,如果相同的问题已经有官方回答了,就不要再去问了,节能减排~\par

最后,如果确认了沟通的必要性,\textbf{请按照申请学校的说明操作}。发邮件的话确定小蜜的邮箱、标题格式和附加信息等。小蜜会收到很多邮件,不满足要求的邮件可能被拒收。\textbf{确保你的邮箱能正常收发邮件},比如院邮好像接收一些学校的邮件就有问题,推荐使用Gmail。如果是打电话,确定该打哪个号码、什么时段可以打。

\subsection{邮件沟通}
由于是和陌生人沟通,要注意保持礼貌,说清楚问题所在。按照邮件的组成来看:
\begin{enumerate}
\item 标题。简要描述自己遇到的问题,比如inquiry about GRE score requirement,question on application status。有的申请系统会分配一个ID给你,有的学校则是单独给你一个ID,总之有的话建议加到标题里,这么做可能会得到优先处理。有的学校可能要求固定的标题格式,请注意。
\item 称呼。一般标准方法是写To whom it may concern:。当然,如果你知道小蜜的名字,也可以使用Dear XXX。比如小蜜叫John Smith,可以说Dear Mr Smith:。
\item 正文。首先说明自己是谁和正在申请的项目。然后用最简单的语言描述自己遇到的问题、失败的尝试,以及期待的回答。建议加上自己的名字、性别、生日以帮助确认身份。如果需要提交一些材料,记得附上附件。
\item 结尾。一般用Best wishes或者Best regards就好。最后要附上自己的联系方式。
\end{enumerate}
\textbf{一个例子}\\
标题:A question about my application status (U-M ID: 123456789)\\
To whom it may concern:\par
I'm applying for UMich's CS master program of 2013 fall. I successfully submitted my application before deadline and have been checking wolverineaccess.umich.edu from time to time.\par

I find that in my credentials, the transcript is still marked as NOT RECEIVED so I am now very anxious about it. Is this situation normal? What should I do?\par

My name is: Jack Ma, birthday: 1990/04/23, U-M ID: 123456789, program: Computer Science \& Engineering M.S.\par

Looking forward to your response.\\

\hspace{-3ex}Best wishes,\\
Jack Ma\\
Software Institute, Nanjing University\\
jackma@gmail.com\par

在邮件发出之后,很多学校设有自动回复,请按照自动回复邮件的内容决定下一步操作。比如有的会明确要求你在标题中加入类似[Second Request]什么的。再比如些学校有单独的ticket系统处理问题,那么请移步其他系统。\par

正常情况下,小蜜会在几个工作日内回复你的邮件。当然,任何事情都有可能发生的,有些学校的小蜜响应速度极慢,甚至从来不回邮件。这种时候,只有:1. 看看邮件是不是被归类到垃圾箱了;2. 从【沟通前的准备开始】再来一次;3. 关门,打电话!;4. 烧香/祈祷。

\subsection{电话沟通}
在申请的时候,由于不少小米邮件不会及时回,为了获得及时的信息更新,打电话是一个比较合适的办法。\par

打电话最实惠的工具当然是Skype。Skype提供了很多种充值方式,有那种一次性的充值卡,也有包月的那种。我用的是后者。26元包月能打美国加拿大200分钟。\par

小米在学校里的职员划分里属于Staff。所以要打电话时要在Staff中找人,千万不要打给Faculty。有些学校会有好几个小米,并且他们都有各自的分工,这里也要注意一下,保证打给对的人,不要浪费双方的时间。\par
     我觉得电话交流中要注意的地方有三点:简洁,清晰,礼貌。 \par 
     1. 简洁。打电话的时候一定不要说废话,否则会让小米非常反感。一般来说,小米接电话时会主动表明身份,如果听到的身份信息没有错,就直接可以开始谈话了。像"Is this the computer science department of XXX University?"这类句子完全可以不用说。另外,尽量要让表达简洁一些,介绍自己时不要说“I am from China and I applied for ..",这样太冗长,直接说"I am a Chinese applicant"即可。\par
     2. 清晰。打电话时一定要把自己的问题说清楚。这也和发邮件类似,发邮件时一般以提的问题为邮件主题,打电话时最好在介绍自己后的第一句话就把问题说清楚。小米每天可能会接上百个来自各个国家的电话,他们没有功夫闲聊。\par
     3. 礼貌。打电话时需要注意一些基本的礼节,这在美国与人交流时需要注意的。在询问时我们可能会常常用"I want to know" , "I wonder"开头,事实上使用"Could you please"这样的句式更有礼貌。另外,谈话结束时要说一句"Have a good day"。

\section{关于套磁}
套磁简单说就是在申请过程中给老师发邮件联系感情以便自己得到录取和奖学金。对于申PhD(和研究型硕士)的同学来说,套磁几乎是必须的,因为你能否被录取很大程度上取决于有没有老师要你,你的RA奖学金也大多直接是你导师给你;但对于申请美国的授课型硕士的同学来说,陶瓷意义不大,因为通常都是由录取委员会根据你的申请材料直接决定,录满为止。\par
\subsection{套磁前}
对于需要陶瓷的同学来说,首先要做的还是明确一个科研方向。这点对于决定读PhD或者研究硕的同学应该不比多说,这个方向应该是你非常感兴趣并且有过一定的相关科研经历或项目经验的。科研方向也应该是你选校(或者选导师)的重要依据之一,并由此获得一份需要套磁的导师名单。\par

大牛采用的最有效也是最佳的陶瓷方式是:自己做一个东西超过你的导师。比如你发现有个老师针对某一问题提出了一个算法模型,你自己花很长时间提出了改进的算法模型得到的实验结果比老师还好。你跟老师邮件里聊这个,老师肯定想要你,因为你证明了他最需要的东西:独立研究能力,进门立刻干活。\par

正常人难以达到这种境界,较多的是聊聊自己的做过的相关项目经验和一些科研想法。如果你在套磁前功课做得足够多:一方面对老师做的东西有一定的认识和自己的见解,另一方面对自己做过的东西能清晰地描述出与老师相关的地方,你套磁就会有起作用。\par

如果没有科研背景或者项目经验,那么只好多读一些相关领域的经典文献,学习一些基本的相关课程,然后找老师最近在顶会上发表的论文了解一下老师近况,然后表示自己对其研究感兴趣。\par

小结一下,套磁前要明确自己科研方向并有所积累,对老师知己知彼。
\subsection{套磁中}
大部分人都是通过邮件套磁,既然涉及到交流,这其中就有很多讲究,比如格式,用词,拼写等等。作为套磁的学生,我们应该时刻提醒自己,“我的字里行间能体现出我是一个怎么样的人”(就像代码一样)。比如,称呼里教授的职称或者姓名搞错,存在一些低级的拼写错误,段落锁紧不一致,空行一会儿多一会儿少……给人的印象就是这个学生不严谨不认真不重视。作为想要套磁成功的你,千万别这么做,其它细节自己也要斟酌好。\par

关于正文,网上搜会有很多模板。这里给出一个简单的模板,是笔者自己使用的,不一定适合你,仅供参考。修改出自己的模板不难,只要问自己教授会对什么感兴趣,你按照合理的结构和逻辑回答就好了。\par
\hspace{-3ex}Example:\\
Hi Professor XX,\\
--个人简介—\\
My name is XX. I am a final-year undergraduate from Software Institute of Nanjing University.\\
--研究经历--\\
I am interested in visual information computing and applications and I have done some works in image processing, including image retargeting and near-duplicated image retrieval. In addition to developing practical software, I am also trying to conduct deep research and publish papers in these areas. I have already one paper accepted by xx , named with xxxxxx (http://.pdf). Besides, I am now preparing for another paper about xx, which will be submitted to xx conference.\\
--为什么去你组里--\\
During my graduate study, I plan to focus on xx in your group, if possible. Your recently published paper about xx fairly interests me. The formulation of the segmentation as an inference of a CRF is very creative and it actually achieved good results (User’s less interaction yet superior segmentation quality). I think doing research like that is very fascinating! I would appreciate it very much if I could have similar great experiences at the start of my research career.\\
--表个决心--\\
Thus, I am eager to do first-class research in your group and I will spare no efforts to conduct projects under your supervision\\

\hspace{-3ex}Kind regards,\\
XX\par
这个水平很一般,相信读者可以从这里找到点感觉写出自己更好的模板。\par

多说两句:关于邮件长度,不要太长,超过300字的你是教授也不会愿意读。关于内容,自我介绍中附上个人主页很方便,附简历的PDF就很不合适了(要慎重);醒目的标出论文(链接)会很加分。关于邮件题目,可选类似于About possible opportunity to study on A (研究方向) and related fields或者更学术化的某个科研问题的探讨。
\subsection{套磁后}
在不确定自己的邮件是不是被老师打开时,等回复比较难受。为了减轻痛苦,需要用到一个叫spypig的东西,把它贴在自己的邮件后面,可以知道教授有没有打开你的邮件,详见\href{http://www.spypig.com/}{这里}。如果人家几天过了都没打开,那你过两天可以再发一封试试;如果打开了却没收到回复,那也别想太多,你只管努力。\par
由于笔者水平有限,关于套磁暂时也只能给出这些基本内容,希望能给没到申请年级的同学们一点点帮助。我深深的感觉到,套磁只是申请众多任务中的一小块,更主要的任务还是切实提高申请人的实力,具体而言就是科研经历,项目经验,论文,系统等等。真心希望各位能合理安排自己的学习生活,努力提高自身实力。在强大的实力后盾下,套磁只是小菜一碟,申请必会马步平川。



