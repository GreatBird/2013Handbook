\chapter{三年你可以做什么}
\newpage
\section{科研}
我从大三上学期结束时开始做研究,到大四上学期开学时论文写作完毕投稿,一共10个月的时间。最后发表在IPDPS 2013的主会议上(接收率19\%,国内算B类偏上的会议,国外一般算并行与分布式方向的顶级会议。今年中国过去发主会议论文的貌似就我一个人),结局算普大喜奔吧~\par

动机和选择。本科生做研究可以选择跟创新项目或者自己找导师跟导师研究组,这个不重要,重要的是你对要做的研究方向或项目是否感兴趣(我的一个朋友做创新项目一直觉得他那项目没什么用,这样结果肯定悲剧)。研究是很辛苦的脑力劳动,有兴趣才有坚持下去的动力。关于导师的选择,有些年轻老师(如我们院的陈振宇老师,强烈推荐),不仅水平杠杠的,还会花很多精力指导学生,跟这样的导师学到的东西是最多的。研究过程中(主动)与导师活跃交流很重要,很多名气大的老师反而忙于奔波没时间跟你讨论。此外,自己的独立思维必须贯穿于整个研究过程。
搞研究的第一个阶段是学习该方向的背景知识:读论文、跟导师讨论、再读论文。一开始导师给了我一篇厚厚的综述论文,熟悉研究的大背景,读完后又给了我一个reading list,我又花了大概三个月读reading list和自己找的相关论文(这一阶段大概看了四五十篇论文)。看论文时必然会产生各种疑问和想法——这个很重要,需要多跟导师交流,最好把所有问题和想法即时记录下来。随着知识的积累理解的深入,大多疑问会陆续消去,剩下的问题都是精华了,可以跟导师详细讨论之。研究问题基本在这过程中产生了。了解背景知识之后,你会对你所研究的东西有更深入的理解,提出好的问题和想法。跟导师讨论时,不能盲目听从导师的结论,因为导师接触你想法的时间毕竟有限,可能理解不到位,所以需要对导师的反馈辩证分析——是你的想法不完善呢,还是导师没有理解你的意图。    一种比较方便的做法就是参加项目,或者导师直接给你Idea。弊端显而易见:你可能无法理解、可能没兴趣、可能认为这玩意儿做了没用,然后做着做着就茫然了没动力了。总之对自己开发的Idea才能理解足够深入、兴趣足够浓厚、应变能力足够强。\par
有了可供研究的想法后,下面就是解决问题了。这一阶段需要根据你要研究的问题,定位并检索到所需知识,然后一边自学相关知识一边讲其运用到你的问题上。这段时间是最耗脑力的,很辛苦(辛苦并快乐着,如果对问题有兴趣的话)。\par
实验是一个体力活,考验你的动手能力。我花2个月待机房做实验搭平台拿数据累死累活的。推荐在这阶段找点帮手,给他挂个作者。\par
最后就是写论文了,主要考验语言表达能力和英语写作功底。其实英语写作不是问题,只要中文表达能力强,英文写作也不会弱,只需要多模仿你读过的论文的写法就行了。论文写作过程一定要导师的参与,导师改出来的论文质量比自己写的一般会高出好几个档次(我的文字表达能力很差,多亏导师给我改了论文)。\par
最后谈一下研究过程中的英文水平问题。计算机专业的研究过程中的所有文献都是英文的,论文写作也是英文,因此英语基础是必要条件。这个问题是很多同学关心的,其实是最没什么好谈的,因为计算机专业的英文文献还是很通俗易懂的,如果实在有困难,多花点时间就解决了(花时间就能解决的问题就不是问题)。


\section{学科竞赛}
学科竞赛是证明自己能力的一个很好的方式。无论是类似Google这种大公司组织的Code Jam亦或是非常具有Academic气息的数学建模大赛,相信在你的CV和PS中,都是值得大书特书的一个章节。本节带来的经验是一位获得数学建模国奖的学长,最后他也获得了UPenn,UW-Madison和UCSD多所学校的AD。

\subsection{数学建模比赛}
我在大三上学期参加的国赛数学建模,最后获得的是全国二等奖。对于国赛数学建模在申请中的作用我不太了解,但是听说美赛数学建模的奖项在申请中有明显积极作用。所以,如果是为了申请,\textbf{建议参加美赛}。\par
我参加的是国赛,所以我说说参加国赛的经验。首先,数学建模是一项三个人的团队竞赛,所以首要任务就是要找好队友。我一开始的队友是同学推荐的,很不靠谱。本来定好了晚上在他们宿舍讨论问题,结果晚上到了那儿连问题描述都没看。好不容易开始讨论了,但是没讨论多久他们却开始玩实况了。遇上这种极品的队友要赶紧退出,退出要趁早。我当时就是退出得晚,那时大家差不多都组好队了,我只是象征性地在培训教室的黑板上留了联系方式,当时心里已经放弃建模了。不过很巧,就在我清理好东西准备离开仙林的时候(建模在仙林培训,当时宿舍已经搬到鼓楼了),竟然有人打电话给我说队里正好缺一个人。加入了之后,发现他们要靠谱多了。\par
其次,在比赛的时候要进行合理的分工。一般来说,三个人的分工是:一个人建模型,这个东西数学系和物理系的比较在行,一个写代码,这个东西是咱软院和计科的饭碗,还有一个主要管理论文的书写,这一部分找商院的最合适。我所在的团队就是按照这样分工的,我觉得这样分工最合理,每个人都可以发挥出自己的长处,能有很高的效率。\par
最后,比赛的时候要注意论文的书写。比赛有三天,我觉得最迟在第三天的下午就要开始写论文。当时我比赛的时候,前面坐的一队显得非常牛B,一直在高谈阔论,一会儿要这个模型,一会儿要那个模型,并且甩出来各种听不懂的词汇。但是直到最终提交前的8个小时他们才开始写论文,最终他们连省三等奖也没拿到。另外,论文要写得专业一些,这也是将来写毕业论文时需要注意的。第一,要消除口语化词汇。“我们”,“本项目”这种第一人称代词尽量不出现,论文中第一人称词汇应该只有“本文”、“本章”、“本节”。第二,要尽量让自己的论文有学术范。数模这么多参赛队,每个队都是超过20页的论文,评委显然不会细看,他们只会粗略地看用到了哪些模型,得到了哪些结果。所以,如果在论文中适当地甩出一些高端专有名词,或者摆出一些看起来很牛B的公式,说不定就能让你的论文在评委眼中脱颖而出。\par
总之,数模是一项很有意义的经历,不管它对申请有用与否。这是一次绝佳的考察跨院系合作能力和实践能力的机会,在这过程中你将学会如何利用已有的知识去解决一些现实的问题。数模带给我的不仅仅是一个全国二等奖,它还让我学会了Matlab,也了解了很多著名的数学模型。所以,如果有充裕时间,我建议去参加一下数学建模。

\section{交换学习}
虽然国外教育鼓励多样性,但是对于教育体制而言,外国高校依然没有那么信任“国货”。本篇和大家聊聊交换学习。
\subsection{短期项目}
许多中介机构纷纷推出这样的1-2个月的游学项目,并经常暗示性地说参加这种短期的游学项目能够帮助申请者提高背景。但是,笔者认为1-2个月的国外学习、生活经历除了帮助你了解国外的学习和生活氛围外,对于个人在申请背景的提高上,帮助意义不大。这些项目就如同学校也会有的短期交流项目一样,仅仅是去玩了一圈而已。\par
首先,首次在国外生活往往需要经历2-3个月的Cultural Shock时期。最开始是对于国外环境的兴奋和好奇,接着因为在生活中遇到的种种不便而又转入对于家的思念,到最后你才能摆脱前面所说的两种情绪,适应在国外的生活,追寻自己的目标。因此从适应的角度上,游学项目因为时间的限制,很难帮助你进入国外的生活环境并结交到对你有帮助的朋友。\par
其次,时间太短也没有办法帮助你取得推荐信。很多希望出国留学的同学都希望能够获得国外老师的推荐信,而一般推荐人应该是对自己有相当了解的人。在网申推荐人系统中,一般都会要求推荐人填写与被推荐人认识多长时间这一项。认识时间越长,推荐人自然和被推荐人更了解。因此,笔者并不推荐玲琅满目的游学项目。\par
撇开游学项目,像Berkeley等学校会设置官方的Summer Session,有兴趣的同学可以去Berkeley官网\href{http://summer.berkeley.edu/}{了解详情},这种短期项目比游学项目要好很多。
\subsection{学校的交换项目}
为了满足越来越多同学走出国门的愿望,学校的交换项目也越来越多。比如笔者申请交换项目那年,学校就有纽约州立石溪分校、加州大学戴维斯分校、加州大学洛杉矶分校以及维斯康辛麦迪逊分校的交换项目,这些交换项目都具有很好的质量。\par
同学们对于学校交换项目的误区还有就是认为竞争会很激烈,肯定轮不上自己。然而,现在学校的很多项目都是一批自费加上少数公费学生的方法来进行筛选的。所以对于家境不是很差的同学,都可以考虑大胆地申请。学校的交换项目第一个好处就是时间长,一般都在4-6个月,有的项目在进入了之后,还可以申请延长交换时间到一年。有了这么长的时间,可以帮助同学们和老师进行深层次的交流,为获取高质量的推荐信奠定基础。此外,可以积极询问参加老师科研项目的机会,不仅仅可以获得推荐信,更能提升自己的CV,并可以当做PS中的素材。时间长还有的好处就是可以在假期去去自己心仪的学校,激发自己申请的动力。同时,自己对于美国社会的了解也会更加深入。\par
\subsection{交换的时机}
每个学院课程安排不同,例如软院一年就安排成了三个小学期,因此很多大三出去交换的同学可能要神奇地避过两个学期的期末考试,这对于希望按时毕业的孩子来说是个挑战。笔者在大三的时候因为考G和课程安排没有选择出国交换,而是看准了舍弃大四上最后一个小学期的课程能够获得交换学习收益的最大化后,在大三下开学的时候申请了大四上出国交换的项目。随后笔者在大四的9-12月在美帝度过了令人兴奋的交换生活,并在美帝完成了自己的大部分申请。\par
回顾自己的申请过程,笔者认为交换学习为自己的背景提升了一个档次。笔者在美帝获得了两份老师的推荐信(其中一份是教授,一份是讲师)、参加了课后的科研项目并完成了项目开发、通过学校的资源完成了自己文书的所有部分。可以说,大四上的交换学习对于笔者的申请很有帮助。当然,最大的收获是拓展了自己生活的宽度。\par
因为每个人的实际情况都不相同,笔者的经验也不能复制给其他同学。笔者只是希望能将自己的经历分享给大家,从而可以帮助大家在背景提升方面获得些启发。

\section{实习}
记得老罗说过他是一个有着工匠情怀的汉子。作为工科生,最大的乐趣应该就是创造。一次实习,能够让你明白象牙塔和实战的区别,能够让你明确自己是否真的喜欢从事这份事业。下面的总结摘自一位实习经验丰富的学长\href{http://blog.sheimi.me/blog/2013/06/17/my-internship.html}{博客}:\par
今天上午终于去公司办完了离职手续,我在奇多6个月的实习也画上了句号。在大四的一年里,我一共参加了两个公司的实习:eBay和奇多信息科技。这两个公司一个是比较成熟的业界大佬,一个是初创的小公司。这一篇文章主要写的是我在这两家公司的实习的经历以及感受。
\subsection{eBay}
eBay就从实习生招聘开始,说实话,eBay的实习生招聘比较坑爹。为什么说坑爹呢,首先,笔试的时候有一小半的题目是英语的阅读理解题。接着,面试的时候尤其是我在的APD部门,内推占了很大的比重。举个例子吧,有个同学技术面的时候面试官的评价是“技术没问题”,英语也没有问题,可是最后被默拒了。不过eBay的员工挺会玩儿的,APD部门还别出心裁然让我们去上海参加了一个Job Fair,目的是通过Job Fair让我们了解APD各个Team的具体工作,然后我们选择两个Team参加面试,再决定最后呆在那个Team里面。\par

eBay的工作环境是很不错的,整个办公室是一个大的开放式的环境,整个部门,从部门老大,到Manager,到实习生都在这里工作,只不过由于位子有限,我们实习生只能够窝在一个小角落里。实习生来的时候可以领到一个旧电脑,我一开始幸运的领导了一台新电脑,但是很快被发现了,又被换成了旧电脑。电脑的系统是企业版windows,但是对于一些喜欢Linux/Unix的人,可以自己换成自己喜欢的操作系统,但是喜欢OS X的就哭吧。eBay是不可以用自己的电脑的,我一开始不知道,拿着自己的电脑用了几个星期,最后被部门老大发现了,只能换回公司配的电脑了。其实eBay这样的公司不允许使用自己的电脑是有原因的,eBay里面有着上亿用户的交易数据,这些数据是用户的隐私,是不能够让我们带走的,听说有一次,公司检测到有个实习生一次把数据用U盘拷走了,后来检测Log的时候发现了,这时那个实习生已经到了美国留学,公司还是紧急联系到他,让他把数据删掉。\par

但是对于eBay这样的公司,我个人感觉它不像一个技术公司。和阿里不一样,eBay的许多关键的技术都不是自己开发出来的,都是从别的地方买来的。比如说,在阿里,他们有自己的TFS,有自己版本的MySQL集群,有自己版本的Nginx服务器,像这些核心的技术,几乎都是自己基于开源技术改进的,而eBay,数据库就使用的Oracle,数据仓库也是用商业的Teradata,Hadoop集群也是使用的社区的发行版,他们所做的工作就是在这些系统之上建立一些应用。并且,在eBay,有些人的工作就是监视运行的任务有没有在规定时间内完成。所以eBay很缺少一些自动化的机制,不过也有一部分人在做自动化的尝试,但是由于这个不是eBay的主营业务,因此开发的进度也很缓慢。\par

不过,在eBay照样可以学到很多的东西。eBay里有很多厉害的人。比如我的Manager就是一个点子特别多得人,常常一拍脑袋就有一个好的想法,我的Team Leader是一个很牛的Linux工程师,看着他满屏卷动的绿色的命令,我看着只有羡慕的份儿了(顺便吐槽一下,满屏的绿色的命令真是亮瞎我的金坷垃狗眼)。并且,eBay有着很严格的软件过程管理,他们使用的是敏捷软件开发中的Scrum,每两周一次迭代。虽然在学校里的实践课里面也能够体会到部分的软件过程,可是总的感觉是照本宣科,为了体验而体验,真正的软件过程只有到了工业界才能够体会得到。还有就是,在eBay的工作语言是英语,虽然平时的日常交流是用中文的(如果Team里面有外国人就另当别论了),但是工作邮件全是英文,有机会还可能参加一些跨国会议,这对英语听说的提高有很大的帮助。\par

在eBay的实习也有让我后悔的事。第一,就是我没经过大家的集体讨论,就替换了整个系统的底层框架,这个对整个Team造成了不小的伤害。说实话,我真不该闲着没事做干这件找抽的事儿。从这件事中也明白了团队的工作和个人的工作是有所不同的。另外就是一开始没搞清楚学校的安排,就出来实习,开学之后就被老师拖回学校上课了,自己的工作也只做了一半就扔下了。

\subsection{奇多科技}

奇多科技是一个小的创业公司,公司现在总共有17个人,其中12个都是南大的实习生,这里面的11个还是我们软院09级的,可以说的南大软院的主场。\par

在创业公司和eBay这样的大公司实习完全有着不同的体验。在这样的创业公司里,完全感受不到自己是一个实习生,有着跟正式员工一样的压力。在这样的一个公司里,几乎放松不下来,开始的几个月,工作时间是从早上9点半到晚上9点,除了吃饭和午睡,几乎全心全意地扑在项目上。虽然我一开始几乎没有接触过Android,但是在开始的一个月之内,写的代码就超过10000行了,这还不包括重构时的代码的删减。在工作的时候,为了保证代码的质量,必须在写代码的时候进行不断的重构,但是同时又要保证开发的进度。虽然压力比较大,但是公司的氛围还是比较好的,正式员工和我们实习生之间几乎感受不到一点隔阂。不过,对于我们的CTO我有着不同的感觉,每一次他从我背后走过的时候,我总是背脊发凉,难道说这就是所谓的“气场”?\par

在这样的创业公司,显而易见,可以学到很多的东西。像我一个Android小白,一开始什么Android的东西都不懂,但是写过一两万行Android代码之后,再怎么搓也是有点感觉的。虽然公司里大多数都是实习生,但是公司的正式员工都是很厉害的人。在与他们平时的交流中、听他们讨论一些问题的时候,或多或少地能够学到一些东西。上面说的有气场的CTO超厉害,在遇到问题的时候上一下厕所,就立马冒出两三个可供选择的解决方案,虽然方案不一定是可行的,但是都是值得去进一步推敲的。\par

这么早就离职我是很不好意思的,因为一开始来实习的时候,我说我会干到8月份,可是想了想大四的暑假还是在家里多陪陪家人,因此就下定决心离职了。不过,今天去公司的时候,公司的HR十分友善,最后还跟我们说,“欢迎随时回公司来玩”,让我感到十分地温暖。

\subsection{总结}

像eBay这样的大公司,主要是团队的合作,在里面活儿挺轻松的。想要水水实习的,去eBay是个不错的选择。不过想要锻炼自己的技术的同学最好还是不要去eBay了,去像百度、腾讯这样的技术公司,对技术上应该会有很大的提升。不过最刺激的还是一些创业小公司,在这些公司里,你可以也是必须独当一面负责你所做的项目,但是不好的地方是它不能够像百度、Google这样大公司一样给你提供一些平台。就比如,像系统底层、基础架构的开发,在这些小公司里几乎是没有办法学到的。这也是我一定要离开的原因之一,我主要的兴趣还是底层的开发和服务器端的开发,Android开发对我来说仅仅是想学习一下。\par

这两次的实习最让我感到失望的是,两次实习的地方都没有严格的代码规范、代码审查和自动化测试。我对代码有一些洁癖,因此,两次实习时我第一个问题就是“代码规范是什么?”,得到的回答都是“没有固定的代码规范”。这让我们合作起来十分麻烦,我很难看懂别人的代码。在奇多,进度的压力使我们没有时间进行代码审查和自动化测试脚本的编写。但是在eBay,代码审查和测试也只是走走过场。\par

这两次的实习都让我学到了不少的东西,生活也十分充实。软件工程确实不是一个在学校里就能够学好的专业,学校里关于工程的东西都只是理论,只有在工业界里不断的体验,才能够有更加深刻的理解。

