\section{蒋苏阳 NJU}
\subsection{写在前面}

今天看到double大神有意搜集每个人的经历经验以供后人参考,我就来实现承诺写一篇攻略。学霸和学神毕竟是少数,我是个学渣,我只希望我的经历能给同想考南大软院研的学渣们一点参考。\par

我9月份从实习岗位回校参加校招,过了大半个月以后觉得当时的我找不到啥好工作,十一回去和家里商量了下决定考研,回校以后在仙林和园租了房子开始看书。剩下的时间大约是三个月。其实这三个月我也没有特别认真,虽然我起得算早,不过坚持不到很晚。其间该打球打球,该看电影看电影,该吃饭出去玩也没落下。\par

 

\subsection{简介}

软院工硕大约是55个名额(下一年有没有变动不予讨论,已扣除保研名额),2013年考试人数200-300,从比例上来看还是不算难的。今年的初试线约是321分,名次为第72名。


四门课分别是 政治100分 英语二100分 数学二150分 专业课150分(软件工程45分+数据结构45分+操作系统35分+计算机网络25分)\par

本学渣大约是60多 60多 110  80多 总分勉强够线  最低线基本不用考虑,因为肯定过的。

 

 

\subsection{我复习了多少}

政治:没有看提纲,几乎没有背诵内容\par

浏览了肖秀荣给的知识点精要,一本小书,没有背诵\par

做了大约10年间的政治真题,没有背诵大题目,只是通篇看了一下\par

肖秀荣冲刺8套卷和终极预测4套卷,选择题除了做,对答案以外还回顾过两遍;大题目有选择地背诵,特别是终级预测4套卷里的大题目,有一道和今年考试的题目完全一样。\par

看了一点点答题指南,没有放在心上\par

 

英语:连一套完整的卷子都没做过\par

      对着书写过两三篇大作文小作文\par

      做了近三年英语一的阅读\par

      单词什么的,没有坚持背下去,掩面\par

 

数学:李永乐的数学复习全书 通篇一遍 所有习题都过了一遍\par

      近13年全部的试题,过了两遍。李永乐的历年试题解析,前一半是整张卷子的,后一半是分类做的。所以过了两遍。\par

      李永乐的400题,感觉略难。\par

      其实我还杂七杂八买了两三本习题,不过后面没来得及做。\par

      我在做复习全书和试题解析的时候把觉得不太会和重要的题目都做过标记,在最后的两天又过了一遍。\par

 

专业课:没什么好说的,看看PPT,就当平常期末考试复习那样吧

 

\subsection{考试感觉}

我感觉政治的选择题出得略刁(虽然我没有对答案,不知道大约是多少分)。大题目我觉得不难,不过毕竟背得少,所以扯得多。扯的时候注意一下条理,说不定能有一些意外的加分。对我这种基本没背诵知识点的人来说60+中规中矩。\par

英语二,不得不说比我想像得简单许多,特别是和英语一比起来。对我这样单词量捉急的人来说,生词居然不太多。做题的时候注意一下,合理分配时间。我记得有两个同学跟我说作文差点没来得及写。\par

数学二,按理说数学二的微积分部分内容比数学一和数学三少。但是由于跟数学一和三比,数学二没有那33分概率统计部分,全加在微积分上面了,所以难度不好说谁更难。我大约有3道大题目的后面一半没有做出来,还有一道觉得算错了,一道选择是猜的,一道填空没有做出来。其他的估计有一点点小错误吧。\par

专业课,40道选择,每门课10道,每道2分。前80分知识点太细碎,我都不记得了。后面7道大题目,二叉树没有写全对,PV操作也没有写对,总得来说专业课答得不好。到后面只能胡扯。

 

\subsection{改进方法和我的建议}

我觉得对政治和英语没什么感觉,学渣表示满足了。如果要说改进一些的话,无非就是英语多做些题目,政治多背诵一些重要内容吧。\par

对于数学,我觉得110总体上可以接受,但我犯了一些小错误。比如说我看书的时候基本上记住了大部分的公式(像泰勒公式之类的其实很好背的),但是我在看到质心,形心公式的时候觉得这个不重要,就没有记,结果考到了(解救办法是求出在当时条件下能求出的所有东西,按分步结分的话说不定就有一些分数了)。所以不要想当然,看书的时候认真一点全面一点,总是好的。我觉得对于数学来说,看错题,回顾做过的题目真的很重要,多看几遍,既强化记忆,又深化理解。\par

对于专业课,我觉得软院自主命题的话,咱平常复习时用的往年试卷和复习资料就很有用了。选择题的知识点很碎,需要浏览各科目的PPT,而大题目范围就相对比较小,我这边不列出来,但是复习的时候很明显的。我就是专业课复习得太少,理解得也不够深,最后连一个简单的PV操作都没写出来。除了软工不太好复习外,另外三门都是比较简单的。\par

 

\subsection{总结}

你如果问考软院研难不难?我觉得不难。当然你要是想考,想清楚以后,越早开始准备肯定越好。\par

你如果问,考研有没有意义,有没有必要?对不起,不在本篇讨论范围之内。\par

 

最后,祝大家好运。