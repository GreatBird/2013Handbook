\chapter*{序言}
  \addcontentsline{toc}{chapter}{序言}
\pagestyle{plain}
这份厚厚的手册记载了09级软院同学飞跃重洋的点滴经历和整个飞越过程中的宝贵经验。真正经历过这一切之后才会发现,出国申请的确不是那么简单,需要你的决心和毅力。需要你背下红宝书上的各种光怪陆离的单词,需要你练习写作各种蹩脚的议论文,需要你循环往复的做听力练习,需要你谨慎耐心的和教授陶瓷\ldots\ldots不过,如果你下定决心一定要出国,那么上面提到的种种困难就显得不那么困难了。\par

作为学姐学长,我们希望能为学弟学妹做点贡献,也希望这样的传统能够一直延续下去。毕业后的出路有很多,工作,保研,考研,出国……没有哪一条路比其他的路更好,选择最适合自己的才是最重要的。如果你碰巧选择了出国读研或读博作为你人生的下一篇章,希望这份手册能为你指明方向。\par

这个社会发展的很快,我们无法揣测到下一年的申请趋势,就我们09级而言,除了一个同学去香港读博士,其余同学都选择了攻读硕士学位,究其原因,可能是近些年美帝或加国的IT行业就业比较红火,大家都冲着读完硕士就业而去的。但是回顾前几届的数据 ,我们不难发现,大部分学长学姐选择出国攻读博士。像这样的变化,也是我们始料不及的。这种申请者向硕士倾斜的现象,将导致硕士申请竞争越来越激烈,原本有实力攻读博士的申请者选择硕士学位,整体拔高了硕士申请的难度。\par

申请工作,万变不离其宗,就是要踏踏实实的搞申请,踏踏实实的学习,提高GPA,踏踏实实的准备GT,踏踏实实的搞研究、实习,争取一封有含金量的推荐信。这些我认为都无法投机取巧的,一切都和踏实努力成正比。手册里所记录的方法和经验,也仅仅能告诉读者,申请时那些工作能为你加分,哪些事情是最好避免的,然而这些经验和技巧都是建立在拥有一个良好的硬件背景的基础上的。所以,与君共勉。\par

祝愿各位在自己选择的路上一帆风顺!\par
\hspace*{\fill} 叶韵致
\clearpage
