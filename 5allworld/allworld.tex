\chapter{Hello, World!}
\newpage
\section{Big Ben}
\section{早安温哥华}
\section{富士山下}
\section{维多利亚湾}
下面是一位申请到香港科技大学CS硕士的学姐经验。

\subsection{总结}
冒出要去香港读研的念头比较晚,整个准备过程都显得有些仓促,也有很多缺陷,不过好坏都是经验,在此与大家分享。\par

组成木桶的木板长短不一,如果用木桶装水,那所装水的体积主要由最短的木板决定,如果用木桶装熟透的米饭,那么可以利用最长的木板来增大木桶所装米饭的体积。在申请过程中,需要突出自己具有的优势,弥补缺陷。我的GPA较低,80出头的成绩刚好能够满足申请香港各所大学研究生的条件,但在众多竞争者中实在没有多少竞争力。较幸运的是,一年半的科研经验提高了自己的硬件条件。在本科阶段,只要曾参与到科研项目中,即使没有发表论文,都应该在简历或者面试中作出适当介绍。例如,在CV中,可以简要陈列自己曾参与的科研项目,简要描述自己在这些项目中作出的主要贡献,将已发表的论文以标准格式列出,并用粗体突出自己的名字。在PS中,除了突出自己申请该学校、该项目的原因外,也可以委婉地描述自己的优点,向学校陈述为什么评委会应该同意你的申请。申请香港大学MPhil/PhD在申请需要写RP,建议在写RP前与自己研究的课题相关的论文,在RP中可以简要描述自己的研究经历、对具体研究课题的认识(优缺点)、提出可扩展的研究方向。\par

\subsection{申请}
香港的研究生分为MSc与MPhil,MPhil需要跟随导师做项目,申请该项目的成功与否与导师的意向有很大关系,对于没有特别优势的自己来说,套磁是一个必要步骤。在确定想要申请的学校之后,还要选择好的导师,可以通过查看导师的主页、联系导师指导过的学生等多个渠道了解导师的研究方向、个性、指导原则等情况,从而从众多教授中选择出需要套磁的对象。一般情况下,可以同时联系多个学校的多个教授,以节省时间,增大几率,但是不建议同时联系同系的不同教授。在对教授进行了解时,需要知道需要的是什么样的学生,有的教授需要的是勤奋型的学生,有的教授需要的是思维灵活的学生,可以根据具体情况突出自己的优点。如果教授已经肯定了申请者的能力,鼓励申请者申请某个项目,那么申请该项目的成功率就近半了。\par

在选择推荐人时,较幸运的是能够找到与自己想要申请的教授相识的人作为推荐人。因为对于教授来说,了解大多数申请者的途径只是通过阅读CV,PS等文书,教授也会怀疑这些文书的可信度,如果这个时候能够有熟人为自己做出推荐,会增加教授对申请者的信任,提高申请的成功几率。\par

无论是CV,PS,RP还是RL,在提交之前都需要从文书组织结构与语言语法两个方面进行重复修改。这个阶段可可以找专业培训机构,或者在国外的朋友帮忙修改。这两个方法我都尝试过,在修改文书之初,我先找了南京某培训机构的一名老师,但是修改后觉得效果欠佳。后来请了一些有申请经历的朋友帮忙,重复修改了很多次,才敢提交申请文书。个人感受,如果能够找到那些了解自己并已经熟悉国外思维模式的朋友帮忙,能够为申请者节省很多精力。当然,申请者自己也需要一遍遍地对文书组织结构进行修改,一是因为申请者应该充分熟悉自己的文书内容,二是最了解申请者经历的人是申请者自己。
